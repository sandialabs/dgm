\documentclass[11pt]{amsart}
\usepackage{geometry}                % See geometry.pdf to learn the layout options. There are lots.
\geometry{letterpaper}                   % ... or a4paper or a5paper or ... 
%\geometry{landscape}                % Activate for for rotated page geometry
%\usepackage[parfill]{parskip}    % Activate to begin paragraphs with an empty line rather than an indent
\usepackage{graphicx}
\usepackage{amssymb}
\usepackage{epstopdf}
\DeclareGraphicsRule{.tif}{png}{.png}{`convert #1 `dirname #1`/`basename #1 .tif`.png}

\title{Local TV Regularization}
\author{S.\ Scott Collis}
\date{\today}

\begin{document}
\maketitle

\section{Background}
The local Total-Variation (TV) regularization is given by
\[ J_{TV}(u) = \alpha \sqrt{ \gamma + \nabla u \cdot \nabla u}  \]
and the gradient is then
\[ J_{TV}'(u) = \alpha \frac{\left( \nabla u \cdot \nabla u' \right)}
{\sqrt{\gamma + \nabla u \cdot \nabla u}} \]
which we can rewrite as
\[ J_{TV}'(u) = \alpha \frac{\left( \nabla u \cdot \nabla u' \right)}
{J_{TV}(u)} \]
If we consider just the numerator and write this in terms of a discrete derivative operator (in one space dimension), then we get
\[ \nabla u \cdot \nabla u' \approx D^h u \cdot D^h u' \]
Which can be written as
\[ D_{ij} u_j D_{in} u'_{n} = K_n u'_{n} \]
which means that we just need to evaluate the elements of the vector 
\[ K_n = D_{ij} u_j D_{in} \]
In multiple space-dimensions, there will be a sum of terms in each coordinate direction.

\section{Implementation}
\begin{enumerate}
\item Evaluate the local derivative operators.  There is an element method to compute these but I'm not sure if its implementation is optimal or correct (especially for general element types).
\item Implement a unit-test to make sure that the local derivative operator is correct by testing the action of this operator on a vector against the element derivative operator.
\item Once this operator is verified, use to correct the \texttt{LocalTV::gradient()} method.
\item While local TV will be an improvement, I believe that we may also need some control over inter-element jumps.  This could be accomplished by adding a second regularization term of the form
\[ J_{JV} = \beta \sqrt{ \gamma + [[u]] \cdot [[u]] } \]
\end{enumerate}

\end{document}  