\documentclass[a4paper,12pt]{article}
\usepackage[utf8]{inputenc}

\usepackage{fullpage}
\usepackage{amsmath,amssymb}
\usepackage{cancel}
\usepackage{amsthm}
\usepackage{stmaryrd}

\title{DGM::Regularization}
\author{Greg von Winckel}
\date{\today}
\begin{document}
\maketitle
\section{Total Variation on elements}
Let the computational domain be denoted as $\Omega\subset \mathbb{R}^n$.
The Sobolev space $W^{1,1}(\Omega)$ denotes the closure of $C_0^1(\Omega)$ with respect to the norm
\begin{equation}
\|u\|_W^{1,1}(\Omega) = \int\limits_{\Omega} |u(x)|+\sum\limits_{i=1}^n 
\Big|\frac{\partial u}{\partial x_i}\Big|\,\mathrm{d}x
\end{equation}

The total variation of a function $u\in L^1(\Omega)$ is defined as 
\begin{equation}
\int\limits_\Omega |Du|=\sup\left\{\int\limits_{\Omega} u\nabla\cdot\varphi\mathrm{d}x\;:\;
\varphi\in C_0^1(\Omega;\mathbb{R}^n)\;:\;\|\varphi\|_{L^\infty(\Omega)}\leq 1\right\}
\end{equation}

Now suppose that the domain is partitioned into a set of nonoverlapping elements

\[
\Omega = \bigcup_k \Omega_k
\]
and let the boundary of the $k$th element be $\partial\Omega_k=\Gamma_k$

Partitioning and integrating by parts, we have

\begin{equation}
-\int\limits_\Omega u\nabla\cdot\varphi\,\mathrm{d}x =
-\sum\limits_k \int\limits_{\Omega_k}u\nabla\cdot\varphi\,\mathrm{d}x
=\sum\limits_k \left(
-\int\limits_{\Gamma_k} f\varphi\cdot\vec\nu\,\mathrm{d}s + \int\limits_{\Omega_k}\varphi\cdot\nabla u\,\mathrm{d}x
\right)
\end{equation}
If we take the supremum over all $\varphi\in[C_0^1(\Omega)]^n$ with $\|\varphi\|_\infty\leq 1$, we obtain the total 
variation
\begin{equation}
\text{TV}[u] = \sum\limits_k\left(\int\limits_{\Gamma_k} |\llbracket u\rrbracket|\,\mathrm{d}s
+\int\limits_{\Omega_k} |\nabla u|\,\mathrm{d}x\right)
\end{equation}
This amounts to summing local $L^1$ norms of the jump on the boundary and the gradient in the interior. This functional
is not differentiable in the classical sense and its gradient consists of subgradients. To avoid this complication,
we can approximate the terms by smooth functions
\begin{equation}
\text{TV}[u] =  \sum\limits_k\left(\int\limits_{\Gamma_k} \sqrt{\llbracket u\rrbracket^2+\alpha} \,\mathrm{d}s
+\int\limits_{\Omega_k} \sqrt{(\nabla u)^2 +\beta }\,\mathrm{d}x\right)
\end{equation}
Where $\alpha$ and $\beta$ are small positive parameters. More generally, we can think of separating this into a surface
and volume penalty term with different weights $\text{TV}[u]=a \text{TV}^S[u] + b \text{TV}^V[u]$ where $a,b>0$.
Let $v\in L^1(\Omega)$ be a test function. We take the 
G\^ateaux derivative of the TV surface functional
\begin{equation}
\lim_{\epsilon\rightarrow 0}\frac{d}{d\epsilon}\text{TV}^S[u+\epsilon v] = 
\lim_{\epsilon\rightarrow 0}\frac{d}{d\epsilon}
\sum\limits_k\int\limits_{\Gamma_k} \sqrt{\llbracket u+\epsilon v\rrbracket^2+\alpha} \,\mathrm{d}s
=
\underbrace{\sum\limits_k\int\limits_{\Gamma_k} \frac{\llbracket u\rrbracket \llbracket v \rrbracket}
{\sqrt{\llbracket u+\epsilon v\rrbracket^2+\alpha}} \,\mathrm{d}s}_{(v,\nabla\text{TV}^S[u])}
\end{equation}
For the volume term we get
\begin{equation}
\lim_{\epsilon\rightarrow 0}\frac{d}{d\epsilon}\text{TV}^V[u+\epsilon v] = 
\lim_{\epsilon\rightarrow 0}\frac{d}{d\epsilon}
\sum\limits_k\int\limits_{\Omega_k} \sqrt{(\nabla u)^2+\beta } \,\mathrm{d}x
=
\underbrace{\sum\limits_k\int\limits_{\Omega_k} \frac{\nabla v\cdot \nabla u}{\sqrt{(\nabla u)^2+\beta }} \,\mathrm{d}x}_{
(v,\nabla\text{TV}^V[u])}
\end{equation}
\section{IPG discretized total variation}
As an alternative, we could also think about computing the gradient for the functional on the whole
domain and then discretizing second. The (smoothed) TV functional on the whole domain is
\begin{equation}
\text{TV}[u] = \int\limits_{\Omega} \sqrt{(\nabla u)^2+\alpha} \,\mathrm{d}x
\end{equation}
Taking the G\^ateux derivative, we obtain 
\begin{equation}
\lim_{\epsilon\rightarrow 0}\frac{\partial}{\partial \epsilon}\text{TV}[u+\epsilon v] =
\lim_{\epsilon\rightarrow 0}\frac{\partial}{\partial \epsilon}
\int\limits_{\Omega} \sqrt{(\nabla u+\epsilon \nabla v)^2+\alpha} \,\mathrm{d}x = 
\int\limits_{\Omega} \frac{\nabla v\cdot\nabla u}
{\sqrt{(\nabla u)^2+\alpha}} \,\mathrm{d}x  
\end{equation}
Integrating by parts we obtain the gradient 
\begin{equation}
\nabla \text{TV}[u] = \begin{cases}
-\nabla\cdot (w\nabla u)&\text{in}\;\Omega\\
w\nabla u\cdot \vec\nu &\text{on}\;\Gamma
\end{cases}\quad\text{where}\;w=\frac{1}{\sqrt{(\nabla u)^2+\alpha}}
\end{equation}
Now we apply the same partitioning as above and integrate against a test function $v$
\begin{align}
\int\limits_\Omega (-\nabla\cdot[w\nabla u])v\,\mathrm{d}x &= 
\sum\limits_k\int\limits_{\Omega_k} (w \nabla u)\cdot \nabla v\,\mathrm{d}x -
\sum\limits_k\int\limits_{\Gamma_k} (w \nabla u)\cdot \vec\nu v\,\mathrm{d}s \\
& = 
\sum\limits_k\int\limits_{\Omega_k} (w \nabla u)\cdot \nabla v\,\mathrm{d}x -
\sum\limits_k\int\limits_{\Gamma_k} \{w\nabla u\cdot\vec \nu\}\llbracket u\rrbracket\,\mathrm{d}s
\end{align}
To this we add two terms to obtain the discrete approximation to the gradient
\begin{equation}
(v,\nabla \text{TV}[u])= 
\sum\limits_k\int\limits_{\Omega_k} (w \nabla u)\cdot \nabla v\,\mathrm{d}x -
\sum\limits_k\int\limits_{\Gamma_k} \{w\nabla u\cdot\vec \nu\}\llbracket v\rrbracket
-\tau  \{w\nabla v\cdot\vec \nu\}\llbracket u\rrbracket -
\sigma_k\llbracket u\rrbracket\llbracket v\rrbracket\,\mathrm{d}s
\end{equation}
Where $\sigma_k$ is a penalty term that is inversely proportional to the volume of element $k$ and $\tau$
is a parameter which specifies the variety of interior penalty Galerkin method (IPG) we have. In particular,
$\tau = -1$ makes the method symmetric. 

%\section{LDG discretized total variation}
%#Another possibility is to use only first derivative terms and define the auxilliary variable $\vec q=\nabla u$
%and have the pair of test functions $(v,\vec\pi)$\footnote{Loosely following the notation in Hesthaven \& Warburton}.
%We return to the gradient from the previous section
%\begin{equation}
%\nabla \text{TV}(u) = \begin{cases}
%-\nabla\cdot (w\nabla u)&\text{in}\;\Omega\\
%w\nabla u\cdot \vec\nu &\text{on}\;\Gamma
%\end{cases}\quad\text{where}\;w=\frac{1}{\sqrt{(\nabla u)^2+\alpha}}
%\end{equation}
%Rewriting this with the auxilliary variable, we get 
%\begin{equation}
%\nabla \text{TV}(u,\vec q) = \begin{cases}
%\begin{pmatrix}
%-\nabla\cdot (w \vec q) \\
%\nabla u - \vec q
%\end{pmatrix} & \text{in}\; \Omega \\
%\quad\quad\vec q \cdot \vec\nu &\text{on}\;\Gamma
%\end{cases}\quad\text{where}\;w=\frac{1}{\sqrt{|\vec q|^2+\alpha}}
%\end{equation}

%\begin{equation}
%\int\limits_{\Omega_k} \vec q\cdot \vec v \,\mathrm{d}x + = 
%\end{equation}



\end{document}
