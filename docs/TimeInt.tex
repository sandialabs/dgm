%==============================================================================
%
%  Time advancement methods for DGM code including Adjoints
%
%  Authors:  S. Scott Collis
%
%  Written:  8-28-2001
%
%  Revised:  8-28-2001
%
%==============================================================================
\documentclass[10pt]{article}
\usepackage{times}
\usepackage{fullpage}
\usepackage{comment}
\usepackage[round,sort&compress]{natbib}
%
%.... PDF links
%
\usepackage[pdfmark,%
            colorlinks=true,%
            breaklinks=true,%
            urlcolor=blue,%
            pdfauthor={S.S.\ Collis},%
            bookmarksopen=false,%
            pdfpagemode=None]{hyperref}
%
%.... TeX macros
%
%   EPSF.TEX macro file:
%
%===============================================================================
%   Special hacked version for my Ph.D. thesis
%
%   Scott Collis
%
%   3-10-97
%===============================================================================
%
%   Written by Tomas Rokicki of Radical Eye Software, 29 Mar 1989.
%   Revised by Don Knuth, 3 Jan 1990.
%   Revised by Tomas Rokicki to accept bounding boxes with no
%      space after the colon, 18 Jul 1990.
%
%   TeX macros to include an Encapsulated PostScript graphic.
%   Works by finding the bounding box comment,
%   calculating the correct scale values, and inserting a vbox
%   of the appropriate size at the current position in the TeX document.
%
%   To use with the center environment of LaTeX, preface the \epsffile
%   call with a \leavevmode.  (LaTeX should probably supply this itself
%   for the center environment.)
%
%   To use, simply say
%   \input epsf           % somewhere early on in your TeX file
%   \epsfbox{filename.ps} % where you want to insert a vbox for a figure
%
%   Alternatively, you can type
%
%   \epsfbox[0 0 30 50]{filename.ps} % to supply your own BB
%
%   which will not read in the file, and will instead use the bounding
%   box you specify.
%
%   The effect will be to typeset the figure as a TeX box, at the
%   point of your \epsfbox command. By default, the graphic will have its
%   `natural' width (namely the width of its bounding box, as described
%   in filename.ps). The TeX box will have depth zero.
%
%   You can enlarge or reduce the figure by saying
%     \epsfxsize=<dimen> \epsfbox{filename.ps}
%   (or
%     \epsfysize=<dimen> \epsfbox{filename.ps})
%   instead. Then the width of the TeX box will be \epsfxsize and its
%   height will be scaled proportionately (or the height will be
%   \epsfysize and its width will be scaled proportiontally).  The
%   width (and height) is restored to zero after each use.
%
%   A more general facility for sizing is available by defining the
%   \epsfsize macro.    Normally you can redefine this macro
%   to do almost anything.  The first parameter is the natural x size of
%   the PostScript graphic, the second parameter is the natural y size
%   of the PostScript graphic.  It must return the xsize to use, or 0 if
%   natural scaling is to be used.  Common uses include:
%
%      \epsfxsize  % just leave the old value alone
%      0pt         % use the natural sizes
%      #1          % use the natural sizes
%      \hsize      % scale to full width
%      0.5#1       % scale to 50% of natural size
%      \ifnum#1>\hsize\hsize\else#1\fi  % smaller of natural, hsize
%
%   If you want TeX to report the size of the figure (as a message
%   on your terminal when it processes each figure), say `\epsfverbosetrue'.
%==============================================================================
%
%  New stuff (SSC)
%
\def\includefigs{\let\ifincfigs=\iftrue}
\def\noincludefigs{\let\ifincfigs=\iffalse}
\includefigs

\newbox\epsfvertlab
\newbox\epsfhorlab
\newbox\epsffiglab

\newdimen\epsfvlabsize
\newdimen\scott
%
%  Define axis and figure labels
%
\def\setvlabel#1{\setbox\epsfvertlab=\vbox{\hbox{#1}}}%
\def\sethlabel#1{\setbox\epsfhorlab=\vbox{\hbox{#1}}}%
\def\figlab#1 #2 #3{\setbox\epsffiglab=\vbox to 0pt{%
\ifvoid\epsffiglab\else\box\epsffiglab\fi\vss\hbox to 0pt{\raise #2 \hbox{\hskip #1 #3}\hss}}}
%
%  Verttex macros
%
\newdimen\fighor
\newdimen\figver
\newbox\rotbox
\long\def\lrlap#1{\hbox to 0pt{#1\hss}}
\long\def\verttex#1#2#3{{\fighor = #1\figver = #2\vbox to \figver{\vss%
\hbox to \fighor{\hfill\hsize=\fighor%
\lrlap{\rotstart{-90 rotate}\vbox to \fighor{#3\vfil}\rotfinish}}}}}
%
%  These three macros are needed to implement verttex
%
\def\dvipsvspec#1{\special{ps:#1}}%  passes #1 verbatim to the output
\def\dvipsrotstart#1{\dvipsvspec{gsave currentpoint currentpoint translate
   #1 neg exch neg exch translate}}% #1 can be any origin-fixing transformation
\def\dvipsrotfinish{\dvipsvspec{currentpoint grestore moveto}}% gets back in synch
%
\def\rotstart#1{\dvipsrotstart{#1}}
\def\rotfinish{\dvipsrotfinish}
%
%  Place the axis labels
%
\def\epsfsetlab{%
\ifvoid\epsfvertlab%
\else%
\verttex{\epsfvlabsize}{\epsfysize}%
{\hbox to \epsfysize{\hss\box\epsfvertlab\hss}}%
\fi%
\ifvoid\epsfhorlab%
\else%
\scott=\epsfxsize%
\advance\scott by \epsfvlabsize%
\rlap{\vtop{\hrule height0pt\hbox to \scott{\hss\box\epsfhorlab\hss}}}%
\fi%
}
%
%  Place the figure labels
%
\def\epsfsetover{\ifvoid\epsffiglab\else\box\epsffiglab\fi}
%
%  Old stuff
%
\newread\epsffilein    % file to \read
\newif\ifepsffileok    % continue looking for the bounding box?
\newif\ifepsfbbfound   % success?
\newif\ifepsfverbose   % report what you're making?
\newdimen\epsfxsize    % horizontal size after scaling
\newdimen\epsfysize    % vertical size after scaling
\newdimen\epsftsize    % horizontal size before scaling
\newdimen\epsfrsize    % vertical size before scaling
\newdimen\epsftmp      % register for arithmetic manipulation
\newdimen\pspoints     % conversion factor
%
\pspoints=1bp          % Adobe points are `big'
\epsfxsize=0pt         % Default value, means `use natural size'
\epsfysize=0pt         % ditto
%
% Original version
%
\def\epsfbox#1{
   \ifvoid\epsfvertlab%
   \else\epsfvlabsize=\ht\epsfvertlab \advance\epsfvlabsize by \dp\epsfvertlab\fi%
   \leavevmode\global\def\epsfllx{72}\global\def\epsflly{72}%
   \global\def\epsfurx{540}\global\def\epsfury{720}%
   \def\lbracket{[}\def\testit{#1}\ifx\testit\lbracket
   \let\next=\epsfgetlitbb\else\let\next=\epsfnormal\fi\next{#1}}%
%
\def\epsfgetlitbb#1#2 #3 #4 #5]#6{\epsfgrab #2 #3 #4 #5 .\\%
   \epsfsetgraph{#6}}%
%
\def\epsfnormal#1{\epsfgetbb{#1}\epsfsetgraph{#1}}%
%
% Special version that set labels over the figure
%
\def\nepsfbox#1{
   \ifvoid\epsfvertlab%
   \else\epsfvlabsize=\ht\epsfvertlab \advance\epsfvlabsize by \dp\epsfvertlab\fi%
   \leavevmode\global\def\epsfllx{72}\global\def\epsflly{72}%
   \global\def\epsfurx{540}\global\def\epsfury{720}%
   \def\lbracket{[}\def\testit{#1}\ifx\testit\lbracket
   \let\next=\epsfgetlitbbo\else\let\next=\epsfnormalo\fi\next{#1}}%
%
\def\epsfgetlitbbo#1#2 #3 #4 #5]#6{\epsfgrab #2 #3 #4 #5 .\\%
   \epsfsetgrapho{#6}}%
%
\def\epsfnormalo#1{\epsfgetbb{#1}\epsfsetgrapho{#1}}%
%
\def\epsfgetbb#1{%
%
%   The first thing we need to do is to open the
%   PostScript file, if possible.
%
\openin\epsffilein=#1
\ifeof\epsffilein\errmessage{I couldn't open #1, will ignore it}\else
%
%   Okay, we got it. Now we'll scan lines until we find one that doesn't
%   start with %. We're looking for the bounding box comment.
%
   {\epsffileoktrue \chardef\other=12
    \def\do##1{\catcode`##1=\other}\dospecials \catcode`\ =10
    \loop
       \read\epsffilein to \epsffileline
       \ifeof\epsffilein\epsffileokfalse\else
%
%   We check to see if the first character is a % sign;
%   if not, we stop reading (unless the line was entirely blank);
%   if so, we look further and stop only if the line begins with
%   `%%BoundingBox:'.
%
          \expandafter\epsfaux\epsffileline:. \\%
       \fi
   \ifepsffileok\repeat
   \ifepsfbbfound\else
    \ifepsfverbose\message{No bounding box comment in #1; using defaults}\fi\fi
   }\closein\epsffilein\fi}%
%
%   Now we have to calculate the scale and offset values to use.
%   First we compute the natural sizes.
%
\def\epsfsetgraph#1{%
   \epsfrsize=\epsfury\pspoints
   \advance\epsfrsize by-\epsflly\pspoints
   \epsftsize=\epsfurx\pspoints
   \advance\epsftsize by-\epsfllx\pspoints
%
%   If `epsfxsize' is 0, we default to the natural size of the picture.
%   Otherwise we scale the graph to be \epsfxsize wide.
%
   \epsfxsize\epsfsize\epsftsize\epsfrsize
   \ifnum\epsfxsize=0 \ifnum\epsfysize=0
      \epsfxsize=\epsftsize \epsfysize=\epsfrsize
%
%   We have a sticky problem here:  TeX doesn't do floating point arithmetic!
%   Our goal is to compute y = rx/t. The following loop does this reasonably
%   fast, with an error of at most about 16 sp (about 1/4000 pt).
% 
     \else\epsftmp=\epsftsize \divide\epsftmp\epsfrsize
       \epsfxsize=\epsfysize \multiply\epsfxsize\epsftmp
       \multiply\epsftmp\epsfrsize \advance\epsftsize-\epsftmp
       \epsftmp=\epsfysize
       \loop \advance\epsftsize\epsftsize \divide\epsftmp 2
       \ifnum\epsftmp>0
          \ifnum\epsftsize<\epsfrsize\else
             \advance\epsftsize-\epsfrsize \advance\epsfxsize\epsftmp \fi
       \repeat
     \fi
   \else\epsftmp=\epsfrsize \divide\epsftmp\epsftsize
     \epsfysize=\epsfxsize \multiply\epsfysize\epsftmp   
     \multiply\epsftmp\epsftsize \advance\epsfrsize-\epsftmp
     \epsftmp=\epsfxsize
     \loop \advance\epsfrsize\epsfrsize \divide\epsftmp 2
     \ifnum\epsftmp>0
        \ifnum\epsfrsize<\epsftsize\else
           \advance\epsfrsize-\epsftsize \advance\epsfysize\epsftmp \fi
     \repeat     
   \fi
%
%  Finally, we make the vbox and stick in a \special that dvips can parse.
%
   \ifepsfverbose\message{#1: width=\the\epsfxsize, height=\the\epsfysize}\fi
   \epsftmp=10\epsfxsize \divide\epsftmp\pspoints
   \epsfsetlab%
   \ifincfigs%
     \vbox to\epsfysize{\vfil\hbox to\epsfxsize{%
        \special{PSfile=#1 llx=\epsfllx\space lly=\epsflly\space
            urx=\epsfurx\space ury=\epsfury\space rwi=\number\epsftmp}%
        \epsfsetover\hfil}}%
   \else%
     \epsfsetover%
     \vbox to\epsfysize{\hrule\vss\hbox to\epsfxsize{\vrule height
                        \epsfysize\hfil\vrule}\vss\hrule}%
   \fi%
\epsfxsize=0pt\epsfysize=0pt}%

%
% Version which sets labels over the graph
%

\def\epsfsetgrapho#1{%
   \epsfrsize=\epsfury\pspoints
   \advance\epsfrsize by-\epsflly\pspoints
   \epsftsize=\epsfurx\pspoints
   \advance\epsftsize by-\epsfllx\pspoints
%
%   If `epsfxsize' is 0, we default to the natural size of the picture.
%   Otherwise we scale the graph to be \epsfxsize wide.
%
   \epsfxsize\epsfsize\epsftsize\epsfrsize
   \ifnum\epsfxsize=0 \ifnum\epsfysize=0
      \epsfxsize=\epsftsize \epsfysize=\epsfrsize
%
%   We have a sticky problem here:  TeX doesn't do floating point arithmetic!
%   Our goal is to compute y = rx/t. The following loop does this reasonably
%   fast, with an error of at most about 16 sp (about 1/4000 pt).
% 
     \else\epsftmp=\epsftsize \divide\epsftmp\epsfrsize
       \epsfxsize=\epsfysize \multiply\epsfxsize\epsftmp
       \multiply\epsftmp\epsfrsize \advance\epsftsize-\epsftmp
       \epsftmp=\epsfysize
       \loop \advance\epsftsize\epsftsize \divide\epsftmp 2
       \ifnum\epsftmp>0
          \ifnum\epsftsize<\epsfrsize\else
             \advance\epsftsize-\epsfrsize \advance\epsfxsize\epsftmp \fi
       \repeat
     \fi
   \else\epsftmp=\epsfrsize \divide\epsftmp\epsftsize
     \epsfysize=\epsfxsize \multiply\epsfysize\epsftmp   
     \multiply\epsftmp\epsftsize \advance\epsfrsize-\epsftmp
     \epsftmp=\epsfxsize
     \loop \advance\epsfrsize\epsfrsize \divide\epsftmp 2
     \ifnum\epsftmp>0
        \ifnum\epsfrsize<\epsftsize\else
           \advance\epsfrsize-\epsftsize \advance\epsfysize\epsftmp \fi
     \repeat     
   \fi
%
%  Finally, we make the vbox and stick in a \special that dvips can parse.
%
   \ifepsfverbose\message{#1: width=\the\epsfxsize, height=\the\epsfysize}\fi
   \epsftmp=10\epsfxsize \divide\epsftmp\pspoints
%   \epsfsetlab%
   \ifincfigs%
     \vbox to\epsfysize{\vfil\hbox to\epsfxsize{%
        \special{PSfile=#1 llx=\epsfllx\space lly=\epsflly\space
            urx=\epsfurx\space ury=\epsfury\space rwi=\number\epsftmp}%
        \epsfsetover\epsfsetlab\hfil}}%
   \else%
     \epsfsetover%
     \vbox to\epsfysize{\hrule\vss\hbox to\epsfxsize{\vrule height
                        \epsfysize\hfil\vrule}\vss\hrule}%
   \fi%
\epsfxsize=0pt\epsfysize=0pt}%

%
%   We still need to define the tricky \epsfaux macro. This requires
%   a couple of magic constants for comparison purposes.
%
{\catcode`\%=12 \global\let\epsfpercent=%\global\def\epsfbblit{%BoundingBox}}%
%
%   So we're ready to check for `%BoundingBox:' and to grab the
%   values if they are found.
%
\long\def\epsfaux#1#2:#3\\{\ifx#1\epsfpercent
   \def\testit{#2}\ifx\testit\epsfbblit
      \epsfgrab #3 . . . \\%
      \epsffileokfalse
      \global\epsfbbfoundtrue
   \fi\else\ifx#1\par\else\epsffileokfalse\fi\fi}%
%
%   Here we grab the values and stuff them in the appropriate definitions.
%
\def\epsfgrab #1 #2 #3 #4 #5\\{%
   \global\def\epsfllx{#1}\ifx\epsfllx\empty
      \epsfgrab #2 #3 #4 #5 .\\\else
   \global\def\epsflly{#2}%
   \global\def\epsfurx{#3}\global\def\epsfury{#4}\fi}%
%
%   We default the epsfsize macro.
%
\def\epsfsize#1#2{\epsfxsize}
%
%   Finally, another definition for compatibility with older macros.
%
\let\epsffile=\epsfbox
%
%  End of epsf macros
%
%=============================================================================
%
%  Macros for defining symbols in text
%
\def\ifspace{\ifcat\issp.\else~\fi}
\def\tspace{\futurelet\issp\ifspace}
\def\a{({\it a\kern 1pt})\tspace}
\def\b{({\it b\kern 1pt})\tspace}
\def\c{({\it c\kern 1pt})\tspace}
\def\d{({\it d\kern 1pt})\tspace}
\def\e{({\it e\kern 1pt})\tspace}
\def\f{({\it f\kern 1pt})\tspace}
\def\g{({\it g\kern 1pt})\tspace}
\def\h{({\it h\kern 1pt})\tspace}
\def\i{({\it i\kern 1pt})\tspace}
\def\j{({\it j\kern 1pt})\tspace}
\def\abc#1{({\it #1\kern 1pt})\tspace}

\newcount\ndots
\def\drawline#1#2{\raise 2.5pt\vbox{\hrule width #1pt height #2pt}}
\def\spacce#1{\hskip #1pt}
%
%.... Line types
%
\def\solid{\hbox{\drawline{24}{.5}} }
\def\bdash{\hbox{\drawline{4}{.5}\spacce{2}}}
\def\dashed{\hbox{\bdash\bdash\bdash\bdash\nobreak} }
\def\bldash{\hbox{\drawline{6}{.5}\spacce{2}}}
\def\ldashed{\hbox{\bldash\bldash\bldash\nobreak} }
\def\bdot{\hbox{\drawline{1}{.5}\spacce{2}}}
\def\dotted{\hbox{\leaders\bdot\hskip 24pt\nobreak} }
\def\chndash{\hbox{\drawline{8.5}{.5}\spacce{2}\drawline{3}{.5}\spacce{2}\drawline{8.5}{.5}}\nobreak }
\def\chndot{\hbox{\drawline{9.5}{.5}\spacce{2}\drawline{1}{.5}\spacce{2}\drawline{9.5}{.5}}\nobreak }
\def\chndotdot{\hbox{\drawline{8}{.5}\spacce{2}\drawline{1}{.5}\spacce{2}\drawline{1}{.5}\spacce{2}\drawline{8}{.5}}\nobreak}
%
%.... Symbols
%
\def\trian{\raise 1.25pt\hbox{$\scriptscriptstyle\triangle$}\nobreak\ }
\def\solidtrian{\raise 1.25pt \hbox to 3bp{\special{" newpath  0 0 moveto 3 0 lineto 1.5 2.598 lineto closepath fill}\hfill}\nobreak\ }
\def\circle{$\circ$\nobreak\ }
\def\scircle{$\bullet$\nobreak\ }
\def\diam{$\diamond$\nobreak\ }
\def\solidcircle{$\bullet$\nobreak\ }
\def\smalltriangle{$\scriptstyle\triangle\textstyle$\nobreak\ }
\def\smallplus{$\scriptstyle + \textstyle$\nobreak\ }
\def\smalltimes{$\scriptstyle\times\textstyle$\nobreak\ }
\def\smallnabla{$\scriptstyle\nabla\textstyle$\nobreak\ }
\def\square{${\vcenter{\hrule height .4pt 
              \hbox{\vrule width .4pt height 3pt \kern 3pt \vrule width .4pt}
   	      \hrule height .4pt}}$\nobreak\ }
\def\solidsquare{${\vcenter{\hrule height 3pt width 3pt}}$\nobreak\ }
\def\plus{\raise 1.6pt \hbox{$\scriptscriptstyle +$}\nobreak\ }
\def\x{\raise 1.25pt \hbox{$\scriptscriptstyle \times$}\nobreak\ }
%
%.... Lines with symbols
%
\def\linecirc{\hbox{\drawline{24}{.5}\kern -14.5pt$\bullet$\kern 12pt}}
\def\dashplus{\hbox{\dashed\kern -19pt\plus\kern 8pt}}
%
%.... Legend macros
%
\def\legendtable#1{\vbox{\baselineskip=10pt\tabskip=0pt\let\\=\cr\halign{\hfil##\hskip 3pt&##\hfil\cr#1\crcr}}}
\def\lllegend#1 #2 #3{\figlab {#1} {#2} {\legendtable{#3}}}
\def\lrlegend#1 #2 #3{\figlab {#1} {#2} {\llap{\legendtable{#3}}}}
\def\ullegend#1 #2 #3{\figlab {#1} {#2} {\vtop{\hrule height 0pt\legendtable{#3}}}}
\def\urlegend#1 #2 #3{\figlab {#1} {#2} {\llap{\vtop{\hrule height 0pt\legendtable{#3}}}}}
%
%  duplex mode
%
\def\twosided{\ifTeXtures\else\special{ps::duplexon}\fi}
\def\onesided{\ifTeXtures\else\special{ps::duplexoff}\fi}
%
%  macros for putting scales on figures
%
\newdimen\xorigon
\newdimen\yorigon
\newdimen\scaleval
\newdimen\scaleorigon

\def\setxscale#1 #2 #3 #4 #5 {%
	\xorigon=#1\yorigon=#3%
	\scaleval=#2\advance\scaleval by -\xorigon%
	\tempdimen=#5 pt\advance\tempdimen by -#4pt%
	\divide\tempdimen by 1000%
	\divide\scaleval by \tempdimen%
	\scaleorigon=-#4pt\divide\scaleorigon by 1000%
	\multiply\scaleorigon by \scaleval}
\def\xtickup#1 #2{\tempdimen=#1pt\divide\tempdimen by 1000%
	\multiply\tempdimen by \scaleval\advance\tempdimen by \scaleorigon%
	\advance\tempdimen by \xorigon%
	\figlab {\tempdimen} {\yorigon} {\vbox {\hbox to 0pt{\hss #2\hss}%
		\baselineskip=8pt\lineskiplimit=-5pt%
		\hbox to 0pt{\hss \vrule height 3pt\hss}}}}
\def\xtickdown#1 #2{\tempdimen=#1pt\divide\tempdimen by 1000%
	\multiply\tempdimen by \scaleval\advance\tempdimen by \scaleorigon%
	\advance\tempdimen by \xorigon%
	\figlab {\tempdimen} {\yorigon} {\vbox to 0pt {\hbox to 0pt{\hss \vrule height 3pt\hss}%
		\nointerlineskip\vskip 3pt%
		\hbox to 0pt{\hss #2\hss}\vss}}}
%
%  Just a place holder
%
\def\nofig#1#2{\leavevmode{\vbox {\hrule \hbox to #1{\vrule height #2 \hfill \vrule} \hrule}} }


\title{Time-Advancement Methods for State and Adjoint}
\author{S.\ Scott Collis}
\date{\today}
\begin{document}

\maketitle

%==============================================================================
%
%==============================================================================
\section*{Introduction}

Consider a state equation of the form
\begin{equation}
\frac{dU(t)}{dt} + f(U(t)) = 0
\end{equation}
for $t\in(0,T]$ and $U(0)=g_{ic}$.  We will consider semi-discrete
approximations to this problem using various time-advancement methods.  The
goal is to cast the semi-discrete unsteady optimization problem in a
convenient form that mimics the continuous formulation as close as possible.

%==============================================================================
%
%==============================================================================
\section*{Continuous Formulation}

Introduce weighting functions $\Psi$ which will play the role of adjoint
variables in the following
\begin{equation}
\Psi \cdot \frac{dU(t)}{dt} + \Psi \cdot f(U(t)) = 0 \qquad 
\forall \Psi\in{\cal W}
\end{equation}
Linearize
\begin{equation}
\Psi \cdot \frac{dU'(t)}{dt} + \Psi \cdot f'( U'(t) ; U(t) ) = 0 
\qquad \forall \Psi\in{\cal W}
\end{equation}
For an Euler--Lagrange identity
\begin{equation}
\Psi \cdot \frac{dU'(t)}{dt} + \Psi \cdot f(U'(t);U(t)) = 
-U' \cdot \frac{d\Psi(t)}{dt} + U' \cdot f^*(\Psi(t);U(t)) +
\left[ \Psi \cdot U' \right]_0^T
\end{equation}

%==============================================================================
%
%==============================================================================
\section*{Trapezoidal}

Let $g_{ic}$ be the initial condition, $g_i$ for $i=0,1,\dots,N$ be the
control,$ \Delta t_i = t_i - t_{i-1}$ for $i=1,\dots,N$, and the semi-discrete
state is given by $U_i$ for $i=0,1,\dots,N$.  The linearized semi-discrete
equation is given by
\begin{equation}
\left[ \begin{array}{ccc} 
I & 0 & 0 \\
-I + \frac{\Delta t_i}{2} f_{i-1} & I + \frac{\Delta t_i}{2} f_i & 0 \\
0 & -I + \frac{\Delta t_N}{2} f_{N-1} & I + \frac{\Delta t_N}{2} f_N
\end{array} \right] 
\left\{\begin{array}{c} U_0 \\ U_i \\ U_N \end{array} \right\} -
\left\{\begin{array}{c} g_{ic} \\ \frac{\Delta t_i}{2}(g_{i-1} + g_i) \\ 
\frac{\Delta t_N}{2}(g_{N-1} + g_N)\end{array} \right\} = 0 
\end{equation}
%
Introduce adjoint variables
\begin{eqnarray}
\left\{\begin{array}{c} \Psi_0 \\ \Psi_i \\\Psi_N \end{array}\right\}
\cdot
\left[ \begin{array}{ccc} 
I & 0 & 0 \\
-I + \frac{\Delta t_i}{2} f_{i-1} & I + \frac{\Delta t_i}{2} f_i & 0 \\
0 & -I + \frac{\Delta t_N}{2} f_{N-1} & I + \frac{\Delta t_N}{2} f_N
\end{array} \right] 
\left\{\begin{array}{c} U_0 \\ U_i \\ U_N \end{array} \right\} - \nonumber\\
\left\{\begin{array}{c} \Psi_0 \\ \Psi_i \\\Psi_N \end{array}\right\}
\cdot
\left\{\begin{array}{c} g_{ic} \\ \frac{\Delta t_i}{2}(g_{i-1} + g_i) \\ 
\frac{\Delta t_N}{2}(g_{N-1} + g_N)\end{array} \right\} = \\
\left\{\begin{array}{c} U_0 \\ U_i \\ U_N \end{array}\right\}
\cdot
\left[ \begin{array}{ccc} 
I & -I + \frac{\Delta t_i}{2} f^*_{0} & 0 \\
0 & I + \frac{\Delta t_i}{2} f^*_i & -I + \frac{\Delta t_{i+1}}{2} f^*_{i} \\
0 &  0 & I + \frac{\Delta t_N}{2} f^*_N
\end{array} \right] 
\left\{\begin{array}{c} \Psi_0 \\ \Psi_i \\ \Psi_N \end{array} \right\} - 
\nonumber\\
\left\{\begin{array}{c} \Psi_0 \\ \Psi_i \\\Psi_N \end{array}\right\}
\cdot
\left\{\begin{array}{c} g_{ic} \\ \frac{\Delta t_i}{2}(g_{i-1} + g_i) \\ 
\frac{\Delta t_N}{2}(g_{N-1} + g_N)\end{array} \right\} 
\end{eqnarray}
From this adjoint operator, it is clear that the Trapezoidal method is not
self-adjoint --- i.e. the discrete adjoint of Trapezoidal is not Trapezoidal
on the continuous adjoint equation.  One can readily see this by looking at
the locations for which $f^*$ is evaluated at a given time-level.  There is
also a discrepency due to the staggered value of $\Delta t_i$ used at each
time level, assuming that $\Delta t_i$ is not constant.

Consider an objective functional that has both distributed (in time) and
terminal observation
\begin{equation}
J_{obs}(U) = \frac{\alpha}{2} \langle (U-\hat U)^2 \rangle_{T} +
       \frac{\omega}{2} \int_0^T \langle (U-\tilde U)^2 \rangle dt
\end{equation}
where $T = t_N$.  For consistency, we will always compute time integrals in
the objective funcitonal using the same time-discretization used to integrate
the state equation.  Thus, we can write a the observation term, discretized
using trapezoidal rule as
\begin{equation} \label{e:objective}
J_{obs}(U) \approx \frac{\alpha}{2} \langle (U-\hat U)^2 \rangle_{t_N} +
\frac{\omega}{2} \sum_{i=1}^{N} \frac{\Delta t_i}{2} 
\left( \langle (U-\tilde U)^2 \rangle_{t_{i-1}} +
\langle (U-\tilde U)^2 \rangle_{t_i} \right) 
\end{equation}
Linearize
\begin{equation}
J'_{obs}(U) \approx \alpha \langle (\bar U-\hat U) U \rangle_{t_N} +
\omega \sum_{i=1}^{N} \frac{\Delta t_i}{2} 
\left( \langle (\bar U-\tilde U) U \rangle_{t_{i-1}} +
\langle (\bar U-\tilde U) U \rangle_{t_i} \right)
\end{equation}
Note that $J'_{obs}(U)$ can be written in the following form
\begin{equation}
J'_{obs}(U) = 
\left\{\begin{array}{c} U_0 \\ U_i \\ U_N \end{array}\right\}
\cdot
\left\{\begin{array}{c} 
\omega\frac{\Delta t_1}{2} (\bar U_0 - \tilde U_0)  \\ 
\omega\frac{(\Delta t_i+\Delta t_{i+1})}{2} (\bar U_i - \tilde U_i) \\ 
\omega\frac{\Delta t_N}{2} (\bar U_N - \tilde U_N) + 
\alpha (\bar U_N - \hat U_N)
\end{array}\right\}
\end{equation}
Adding and subtracting this term yields an adjoint equation of the form
\begin{eqnarray}
\left\{\begin{array}{c} U_0 \\ U_i \\ U_N \end{array}\right\}
\cdot
\left[ \begin{array}{ccc} 
I & -I + \frac{\Delta t_i}{2} f^*_{0} & 0 \\
0 & I + \frac{\Delta t_i}{2} f^*_i & -I + \frac{\Delta t_{i+1}}{2} f^*_{i} \\
0 &  0 & I + \frac{\Delta t_N}{2} f^*_N
\end{array} \right] 
\left\{\begin{array}{c} \Psi_0 \\ \Psi_i \\ \Psi_N \end{array} \right\} - 
\nonumber \\
\left\{\begin{array}{c} U_0 \\ U_i \\ U_N \end{array}\right\}
\cdot
\left\{\begin{array}{c} 
\omega\frac{\Delta t_1}{2} (\bar U_0 - \tilde U_0)  \\ 
\omega\frac{(\Delta t_i+\Delta t_{i+1})}{2} (\bar U_i - \tilde U_i) \\ 
\omega\frac{\Delta t_N}{2} (\bar U_N - \tilde U_N) +
\alpha (\bar U_N - \hat U_N)
\end{array}\right\} = 0
\end{eqnarray}
which must hold for all $U\in{\cal V}$.  Thus the adjoint equation in strong
form is simply
\begin{eqnarray}
\left[ \begin{array}{ccc} 
I & -I + \frac{\Delta t_i}{2} f^*_{0} & 0 \\
0 & I + \frac{\Delta t_i}{2} f^*_i & -I + \frac{\Delta t_{i+1}}{2} f^*_{i} \\
0 &  0 & I + \frac{\Delta t_N}{2} f^*_N
\end{array} \right] 
\left\{\begin{array}{c} \Psi_0 \\ \Psi_i \\ \Psi_N \end{array} \right\} - 
\left\{\begin{array}{c} 
\omega\frac{\Delta t_1}{2} (\bar U_0 - \tilde U_0)  \\ 
\omega\frac{(\Delta t_i+\Delta t_{i+1})}{2} (\bar U_i - \tilde U_i) \\ 
\omega\frac{\Delta t_N}{2} (\bar U_N - \tilde U_N) +
\alpha (\bar U_N - \hat U_N)
\end{array}\right\} = 0
\end{eqnarray}
This yields the gradient equation
\begin{eqnarray}
\left\{\begin{array}{c} U_0 \\ U_i \\ U_N \end{array}\right\}
\cdot
\left\{\begin{array}{c} 
\omega\frac{\Delta t_1}{2} (\bar U_0 - \tilde U_0)  \\ 
\omega\frac{(\Delta t_i+\Delta t_{i+1})}{2} (\bar U_i - \tilde U_i) \\ 
\omega\frac{\Delta t_N)}{2} (\bar U_N - \tilde U_N) +
\alpha (\bar U_N - \hat U_N)
\end{array}\right\} =
\left\{\begin{array}{c} \Psi_0 \\ \Psi_i \\\Psi_N \end{array}\right\}
\cdot
\left\{\begin{array}{c} g_{ic} \\ \frac{\Delta t_i}{2}(g_{i-1} + g_i) \\ 
\frac{\Delta t_N}{2}(g_{N-1} + g_N)\end{array} \right\}
\end{eqnarray}
From the gradient equation, we see that due to the manner in which controls are
coupled in Trapezoidal integration, it is a linear combination of adjoint
quantities that yeilds the gradient for each control.
\begin{eqnarray} \label{e:trap_grad}
\left\{\begin{array}{c} \nabla_{g_0} J \\ \nabla_{g_i} J \\ \nabla_{g_N} J 
\end{array}\right\}
= \left\{\begin{array}{c} 
\frac{\Delta t_1}{2} \Psi_1 \\ 
\frac{(\Delta t_i \Psi_i + \Delta t_{i+1} \Psi_{i+1} )}{2} \\ 
\frac{\Delta t_N}{2} \Psi_N 
\end{array}\right\}
\end{eqnarray}
while the gradient with respect to the initial condition is given by
\begin{equation}
\nabla_{g_{ic}} J = \Psi_0
\end{equation}
In most of our applications, we are not interested in $\nabla_{g_{ic}} J$ so
that we can take the $\Psi_i$ vector and convert it to the $\nabla_g J$ vector.

During computation, just compute, extract and place the adjoint information
used to compute the gradient in the gradient vector.  Then one needs to just
go through, from $i=0,\dots,N$, and process the gradient vector using
(\ref{e:trap_grad}) which can be rewritten as
\begin{eqnarray}
\left\{\begin{array}{c} 
\nabla_{g_{ic}} J \\ \nabla_{g_0} J \\ \nabla_{g_i} J \\ \nabla_{g_N} J 
\end{array}\right\} = 
\left[ \begin{array}{ccc} 
1 & 0 & 0 \\
0 & \frac{\Delta t_1}{2} & 0 \\
0 & \frac{\Delta t_i}{2} & \frac{\Delta t_{i+1}}{2} \\
0 & 0 & \frac{\Delta t_N}{2}
\end{array} \right] 
\left\{\begin{array}{c} \Psi_0 \\ \Psi_i \\\Psi_N \end{array}\right\}
\end{eqnarray}
for $i=1,\dots,N-1$.

\subsection*{Add an artifical end state}
The advantage of this formulation is that one explicitly sees the impact of
the initial and end conditions just as in the continuous formulation.  In
other words, we obtain the time-boundary term explicily.  To help in the
implementation, the adjoint variables are shifted by -1 compared to the
previous section.
\begin{eqnarray}
\left\{\begin{array}{c} 
\Lambda \\ \Psi_{i-1} \\\Psi_{N-1} \\ \Psi_N 
\end{array}\right\}
\cdot
\left[ \begin{array}{cccc} 
I & 0 & 0 & 0 \\
-I + \frac{\Delta t_i}{2} f_{i-1} & I + \frac{\Delta t_i}{2} f_i & 0 & 0 \\
0 & -I + \frac{\Delta t_N}{2} f_{N-1} & I + \frac{\Delta t_N}{2} f_N & 0 \\
0 & 0 & -I & I  
\end{array} \right] 
\left\{\begin{array}{c} U_0 \\ U_i \\ U_N \\ Y \end{array} \right\} - 
\nonumber\\
\left\{\begin{array}{c} 
\Lambda \\ \Psi_{i-1} \\ \Psi_{N-1} \\ \Psi_N \end{array}\right\}
\cdot
\left\{\begin{array}{c} g_{ic} \\ \frac{\Delta t_i}{2}(g_{i-1} + g_i) \\ 
\frac{\Delta t_N}{2}(g_{N-1} + g_N) \\ 0 \end{array} \right\} = \\
\left\{\begin{array}{c} U_0 \\ U_i \\ U_N \\ Y \end{array}\right\}
\cdot
\left[ \begin{array}{cccc} 
I & -I + \frac{\Delta t_i}{2} f^*_{0} & 0 & 0 \\
0 & I + \frac{\Delta t_i}{2} f^*_i & -I + \frac{\Delta t_{i+1}}{2} f^*_{i} &0\\
0 &  0 & I + \frac{\Delta t_N}{2} f^*_N & -I \\
0 &  0 & 0 & I \\
\end{array} \right] 
\left\{\begin{array}{c} 
\Lambda \\ \Psi_{i-1} \\ \Psi_{N-1} \\ \Psi_N 
\end{array} \right\} - 
\nonumber\\
\left\{\begin{array}{c} 
\Lambda \\ \Psi_{i-1} \\\Psi_{N-1} \\ \Psi_N 
\end{array}\right\}
\cdot
\left\{\begin{array}{c} g_{ic} \\ \frac{\Delta t_i}{2}(g_{i-1} + g_i) \\ 
\frac{\Delta t_N}{2}(g_{N-1} + g_N) \\ 0 \end{array} \right\} 
\end{eqnarray}
where $i=1,\dots,N-1$.  So the adjoint equation becomes
\begin{eqnarray}
\left[ \begin{array}{cccc} 
I & -I + \frac{\Delta t_i}{2} f^*_{0} & 0 & 0 \\
0 & I + \frac{\Delta t_i}{2} f^*_i & -I + \frac{\Delta t_{i+1}}{2} f^*_{i} &0\\
0 &  0 & I + \frac{\Delta t_N}{2} f^*_N & -I \\
0 &  0 & 0 & I \\
\end{array} \right] 
\left\{\begin{array}{c} 
\Lambda \\ \Psi_{i-1} \\ \Psi_{N-1} \\ \Psi_N
\end{array} \right\} - 
\left\{\begin{array}{c} 
\omega\frac{\Delta t_1}{2} (\bar U_0 - \tilde U_0)  \\ 
\omega\frac{(\Delta t_i+\Delta t_{i+1})}{2} (\bar U_i - \tilde U_i) \\ 
\omega\frac{\Delta t_N}{2} (\bar U_N - \tilde U_N) \\
\alpha (\bar U_N - \hat U_N)
\end{array}\right\} = 0
\end{eqnarray}
In the semi-discrete (trapezoidal) setting, the time-boundary term takes the
form\[ \left(\Psi_N - \alpha(\bar U_N - \hat U_N)\right) Y - \Lambda \left( U_0
- g_{ic} \right) \] which allows one to directly see the dependence on the
initial and end conditions.

Thus, you can still think of terminal observation as an end condition, but, if
you do so, then there are too many adjoint variables.  However, $\Lambda$ is
only used to give the sensitivity to an initial condition, likewise, $\Psi_N$
is actually unneeded as it does not influence the gradient with respect to
$g$.

In this formulation, the gradient is given by
\begin{eqnarray}
\left\{\begin{array}{c} 
\nabla_{g_{ic}} \\ \nabla_{g_0} J \\ \nabla_{g_i} J \\ \nabla_{g_N} J 
\end{array}\right\} = 
\left[ \begin{array}{ccc} 
1 & 0 & 0 \\
0 & \frac{\Delta t_1}{2} & 0 \\
0 & \frac{\Delta t_{i}}{2} & \frac{\Delta t_{i+1}}{2} \\
0 & 0 & \frac{\Delta t_N}{2}
\end{array} \right] 
\left\{\begin{array}{c} 
\Lambda \\ \Psi_{i-1} \\ \Psi_{N-1} 
\end{array}\right\}
\end{eqnarray}
for $i=1,\dots,N-1$.

I would store $\Psi_i$ for $i=0,\dots,N$ although for the gradient, only
values from $\i=0,\dots,N-1$ would be used.  Since {\tt computeGrad} is
currently called in {\tt prestep}, I need to make sure that I call it after
completing the last time step.

\subsection*{Implementation}

After much thought, I think that I prefer the first approach. All that needs
to be done is to initially solve an auxilary problem for $\Psi_N$ given the
end-condition.

%==============================================================================
%
%==============================================================================
\section*{Backward Euler}

Let $g_{ic}$ be the initial condition, $g_i$ for $i=1,\dots,N$ be the
controls, $\Delta t_i = t_i - t_{i-1}$ for $i=1,\dots,N$, and the
semi-discrete state is given by $U_i$ for $i=0,1,\dots,N$.  The linearized
semi-discrete equation is given by
\begin{equation}
\left[ \begin{array}{ccc} 
I & 0 & 0 \\
-I & I + \Delta t_i f_i & 0 \\
0 & -I & I + \Delta t_N f_N
\end{array} \right] 
\left\{\begin{array}{c} U_0 \\ U_i \\ U_N \end{array} \right\} -
\left\{\begin{array}{c} 
g_{ic} \\ \Delta t_i g_i \\ \Delta t_N g_N\end{array} \right\} = 0 
\end{equation}
%
Introduce adjoint variables
\begin{eqnarray}
\left\{\begin{array}{c} \Psi_0 \\ \Psi_i \\\Psi_N \end{array}\right\}
\cdot
\left[\begin{array}{ccc} 
I & 0 & 0 \\
-I & I + \Delta t_i f_i & 0 \\
0 & -I & I + \Delta t_N f_N
\end{array}\right] 
\left\{\begin{array}{c} U_0 \\ U_i \\ U_N \end{array} \right\} -
\left\{\begin{array}{c} \Psi_0 \\ \Psi_i \\\Psi_N \end{array}\right\}
\cdot
\left\{\begin{array}{c} 
g_{ic} \\ \Delta t_i g_i \\ \Delta t_N g_N \end{array} \right\} =
\nonumber \\
\left\{\begin{array}{c} U_0 \\ U_i \\ U_N \end{array}\right\}
\cdot
\left[ \begin{array}{ccc} 
I & -I & 0 \\
0 & I + \Delta t_i f^*_i & -I \\
0 &  0 & I + \Delta t_N f^*_N
\end{array} \right] 
\left\{\begin{array}{c} \Psi_0 \\ \Psi_i \\ \Psi_N \end{array} \right\} - 
\left\{\begin{array}{c} \Psi_0 \\ \Psi_i \\\Psi_N \end{array}\right\}
\cdot
\left\{\begin{array}{c} g_{ic} \\ \Delta t_i g_i \\ \Delta t_N g_N
\end{array} \right\}
\end{eqnarray}
From this adjoint operator, it is clear that Backward Euler is self-adjoint
--- i.e. the discrete adjoint of Backward Euler is Backward Euler applied to
the continuous adjoint equation.  The only potential impediment is in the
indexing of the adjoint solution. Similar to the Trapezoidal method
investigated above, there is an ``off-by-one'' issue.

Using the objective function (\ref{e:objective}), but now integrating using
Backward Euler yields a discretized, linearized objective function of the form
\begin{equation}
J'_{obs}(U) \approx \alpha \langle (\bar U-\hat U) U \rangle_{t_N} +
\omega \sum_{i=1}^{N} \Delta t_i \langle (\bar U-\tilde U) U \rangle_{t_i} 
\end{equation}
Note that $J'_{obs}(U)$ can be written in the following form
\begin{equation}
J'_{obs}(U) = 
\left\{\begin{array}{c} U_0 \\ U_i \\ U_N \end{array}\right\}
\cdot
\left\{\begin{array}{c} 
0 \\ 
\omega\Delta t_i(\bar U_i - \tilde U_i) \\ 
\omega\Delta t_N(\bar U_N - \tilde U_N) + \alpha (\bar U_N - \hat U_N)
\end{array}\right\}
\end{equation}
Thus, the adjoint equation is
\begin{eqnarray}
\left\{\begin{array}{c} U_0 \\ U_i \\ U_N \end{array}\right\}
\cdot
\left[ \begin{array}{ccc} 
I & -I & 0 \\
0 & I + \Delta t_i f^*_i & -I \\
0 &  0 & I + \Delta t_N f^*_N
\end{array} \right] 
\left\{\begin{array}{c} \Psi_0 \\ \Psi_i \\ \Psi_N \end{array} \right\} - 
\left\{\begin{array}{c} U_0 \\ U_i \\ U_N \end{array}\right\}
\cdot
\left\{\begin{array}{c} 
0 \\ 
\omega\Delta t_i (\bar U_i - \tilde U_i) \\ 
\omega\Delta t_N (\bar U_N - \tilde U_N) + \alpha (\bar U_N - \hat U_N)
\end{array}\right\} = 0
\end{eqnarray}
which must hold for all $U\in{\cal V}$.  Thus the adjoint equation in strong
form is simply
\begin{eqnarray}
\left[ \begin{array}{ccc} 
I & -I & 0 \\
0 & I + \Delta t_i f^*_i & -I \\
0 &  0 & I + \Delta t_N f^*_N
\end{array} \right] 
\left\{\begin{array}{c} \Psi_0 \\ \Psi_i \\ \Psi_N \end{array} \right\} - 
\left\{\begin{array}{c} 
0 \\ 
\omega\Delta t_i (\bar U_i - \tilde U_i) \\ 
\omega\Delta t_N (\bar U_N - \tilde U_N) + \alpha (\bar U_N - \hat U_N)
\end{array}\right\} = 0
\end{eqnarray}
and the resulting gradient equation is
\begin{eqnarray}
\left\{\begin{array}{c} U_0 \\ U_i \\ U_N \end{array}\right\}
\cdot
\left\{\begin{array}{c} 
0 \\ 
\omega\Delta t_i (\bar U_i - \tilde U_i) \\ 
\omega\Delta t_N (\bar U_N - \tilde U_N) + \alpha (\bar U_N - \hat U_N)
\end{array}\right\}
=
\left\{\begin{array}{c} \Psi_0 \\ \Psi_i \\\Psi_N \end{array}\right\}
\cdot
\left\{\begin{array}{c} 
g_{ic} \\ \Delta t_i g_i \\ \Delta t_N g_N \end{array} \right\}
\end{eqnarray}
From the gradient equation, we see that for Backward Euler there is a direct
relationship between the adjoint variables and the gradient of the objective
function with respect to the control
\begin{eqnarray} \label{e:backward_euler_grad}
\left\{\begin{array}{c} 
\nabla_{g_{ic}} \\ \nabla_{g_1} J \\ \nabla_{g_i} J \\ \nabla_{g_N} J 
\end{array}\right\}
= \left\{\begin{array}{c} 
\Psi_0 = \Psi_1 \\
\Delta t_1 \Psi_1 \\ 
\Delta t_{i} \Psi_{i} \\ 
\Delta t_N \Psi_N 
\end{array}\right\}
\end{eqnarray}
where $i=2,\dots,N-1$.

\subsection*{Add an artificial end state}

Similar to Trapezoidal, we can shift the adjoint variables so that
\begin{eqnarray}
&\left\{\begin{array}{c} \Psi_0^- \\ \Psi_0 \\ \Psi_{i-1} \\ 
\Psi_{N-1} \\ \Psi_N^- 
\end{array}\right\}
\cdot
\left[\begin{array}{ccccc} 
I & 0 & 0 & 0 & 0 \\
-I & I + \Delta t_1 f_1 & 0 & 0 & 0 \\
0 & -I & I + \Delta t_i f_i & 0 & 0 \\
0 & 0 & -I & I + \Delta t_N f_N & 0 \\
0 & 0 & 0 & -I & I \\
\end{array}\right] 
\left\{\begin{array}{c} 
U_0^- \\ U_1 \\ U_i \\ U_N \\ U_N^+ 
\end{array} \right\} -
\left\{\begin{array}{c} 
\Psi_0^- \\ \Psi_0 \\ \Psi_{i-1} \\ \Psi_{N-1} \\ \Psi_N^- 
\end{array}\right\}
\cdot
\left\{\begin{array}{c} 
g_{ic} \\ \Delta t_1 g_1 \\ \Delta t_i g_i \\ \Delta t_N g_N \\ 0 
\end{array} \right\} 
& =
\nonumber \\
&\left\{\begin{array}{c} 
U_0^- \\ U_1 \\ U_i \\ U_N \\ U_N^+ 
\end{array} \right\}
\cdot
\left[ \begin{array}{ccccc} 
I & -I & 0 & 0 & 0 \\
0 & I + \Delta t_1 f^*_1 & -I & 0 & 0 \\
0 & 0 & I + \Delta t_i f^*_i & -I & 0 \\
0 & 0 & 0 & I + \Delta t_N f^*_N & -I \\
0 & 0 & 0 & 0 & I \\
\end{array} \right] 
\left\{\begin{array}{c} 
\Psi_0^- \\ \Psi_0 \\ \Psi_{i-1} \\ \Psi_{N-1} \\ \Psi_N^- 
\end{array}\right\}
-
\left\{\begin{array}{c} 
\Psi_0^- \\ \Psi_0 \\ \Psi_{i-1} \\ \Psi_{N-1} \\ \Psi_N^- 
\end{array}\right\}
\cdot
\left\{\begin{array}{c} g_{ic} \\ \Delta t_1 g_1 \\ \Delta t_i g_i \\ 
\Delta t_N g_N \\ 0 
\end{array} \right\} &
\end{eqnarray}
Using the objective function (\ref{e:objective}), but now integrating using
Backward Euler yields a discretized, linearized objective function of the form
\begin{equation}
J'_{obs}(U) = 
\left\{\begin{array}{c} U_0^- \\ U_1 \\ U_i \\ U_N \\ U^+_N \end{array}\right\}
\cdot
\left\{\begin{array}{c} 
0 \\
\omega\Delta t_1(\bar U_1 - \tilde U_1) \\ 
\omega\Delta t_i(\bar U_i - \tilde U_i) \\ 
\omega\Delta t_N(\bar U_N - \tilde U_N) \\
\alpha (\bar U_N - \hat U_N)
\end{array}\right\}
\end{equation}
So, the adjoint equation is
\begin{eqnarray}
\left[ \begin{array}{ccccc} 
I & -I & 0 & 0 & 0 \\
0 & I + \Delta t_1 f^*_1 & -I & 0 & 0 \\
0 & 0 & I + \Delta t_i f^*_i & -I & 0 \\
0 & 0 & 0 & I + \Delta t_N f^*_N & -I \\
0 & 0 & 0 & 0 & I \\
\end{array} \right] 
\left\{\begin{array}{c} 
\Psi_0^- \\ \Psi_0 \\ \Psi_{i-1} \\ \Psi_{N-1} \\ \Psi_N^- 
\end{array}\right\}
=
\left\{\begin{array}{c} 
0 \\
\omega\Delta t_1(\bar U_1 - \tilde U_1) \\ 
\omega\Delta t_i(\bar U_i - \tilde U_i) \\ 
\omega\Delta t_N(\bar U_N - \tilde U_N) \\
\alpha (\bar U_N - \hat U_N)
\end{array}\right\}
\end{eqnarray}
And the resulting gradient equation
\begin{equation}
\left\{\begin{array}{c} U_0^- \\ U_1 \\ U_i \\ U_N \\ U^+_N \end{array}\right\}
\cdot
\left\{\begin{array}{c} 
0 \\
\omega\Delta t_1(\bar U_1 - \tilde U_1) \\ 
\omega\Delta t_i(\bar U_i - \tilde U_i) \\ 
\omega\Delta t_N(\bar U_N - \tilde U_N) \\
\alpha (\bar U_N - \hat U_N)
\end{array}\right\}
=
\left\{\begin{array}{c} 
\Psi_0^- \\ \Psi_0 \\ \Psi_{i-1} \\ \Psi_{N-1} \\ \Psi_N^- 
\end{array}\right\}
\cdot
\left\{\begin{array}{c} g_{ic} \\ \Delta t_1 g_1 \\ \Delta t_i g_i \\ 
\Delta t_N g_N \\ 0 
\end{array} \right\}
\end{equation}
So that the gradient of the objective function with respect to the control is
\begin{eqnarray} \label{e:backward_euler_grad_2}
\left\{\begin{array}{c} 
\nabla_{g_{ic}} J \\ \nabla_{g_1} J \\ \nabla_{g_i} J \\ \nabla_{g_N} J 
\end{array}\right\}
= \left\{\begin{array}{c}
\Psi_0^- = \Psi_0 \\ 
\Delta t_1 \Psi_0 \\ 
\Delta t_{i} \Psi_{i-1} \\ 
\Delta t_N \Psi_{N-1} 
\end{array}\right\}
\end{eqnarray}
where $i=2,\dots,N-1$.

\subsection*{Implementation}

Currently in the State, I set the IC corresponding to $i=0$ and then solve the
State equation for $i=1,\dots,N$.

For the Adjoint, I need to set the EC corresponding to $i=N$ and then solve
the Adjoint equation for $i=N-1,\dots,0$.  However, solving for $\Psi_i$
requires that you use State and Control information from $i+1$.  Likewise, the
gradient contribution is also at $i+1$.

Or, $lstep=N,\dots,1$ and we solve for $\Psi_{lstep-1}$ which requires State
and Control information from $lstep$ and the Gradient is stored at $lstep$.

%==============================================================================
%
%==============================================================================
\section*{Discontinuous Galerkin in Time}

\begin{equation}
W \cdot \frac{dU(t)}{dt} + W \cdot f(U(t)) = 0 \qquad 
\forall \Psi\in{\cal W}
\end{equation}
Linearize
\begin{equation}
W \cdot \frac{dU'(t)}{dt} + W \cdot f'( U'(t) ; U(t) ) = 0 
\qquad \forall \Psi\in{\cal W}
\end{equation}
Form an Euler--Lagrange identity
\begin{eqnarray}
\int_{t_i}^{t_{i+1}} W \cdot \frac{dU'(t)}{dt} dt + 
\int_{t_i}^{t_{i+1}} W \cdot f(U'(t);U(t)) dt = \nonumber\\
\int_{t_i}^{t_{i+1}} -U' \cdot \frac{dW(t)}{dt} + 
\int_{t_i}^{t_{i+1}} U' \cdot f^*(W(t);U(t)) +
\left[ W \cdot U' \right]_{t_i}^{t_{i+1}} + 
\left[ W \cdot \widehat U'(U^-,U^+) - U' \right]_{t_i}^{t_{i+1}}
\end{eqnarray}
where we have introduced an upwind numerical flux $\widehat U'(U^-,U^+) =
U^-$.  Thus,
\begin{eqnarray}
\int_{t_i}^{t_{i+1}} W \cdot \frac{dU'(t)}{dt} dt + 
\int_{t_i}^{t_{i+1}} W \cdot f(U'(t);U(t)) dt = \nonumber\\
\int_{t_i}^{t_{i+1}} -U' \cdot \frac{dW(t)}{dt} + 
\int_{t_i}^{t_{i+1}} U' \cdot f^*(W(t);U(t)) +
\left[ W \cdot \widehat U'(U^-,U^+)\right]_{t_i}^{t_{i+1}}
\end{eqnarray}
If we use piecewise constants for our trial and test function spaces, then
this simplifies to
\begin{eqnarray}
W_{i+1}^- \int_{t_i}^{t_{i+1}} \cdot f(U'(t);U(t)) dt = 
\int_{t_i}^{t_{i+1}} U_{i}^+ \cdot f^*(W(t);U(t)) +
( W_{i+1}^- \cdot U_{i+1}^- - W_{i}^+ \cdot U_{i}^- )
\end{eqnarray}

%==============================================================================
%
%==============================================================================
\section*{RK-4}

\subsection*{State}

\[ dU/dt + F(U) = 0 \]
\[ R[0] = F(U) \]
\[ U_t = U - a[0] dt R[0] \]
\[ R[0] = -b[0] R[0] \]

\subsection*{Adjoint}

%==============================================================================
%
%==============================================================================

\end{document}
