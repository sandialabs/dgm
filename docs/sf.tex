

%% 
%% This is a LaTeX document generated by automatic conversion of a
%% Mathematica notebook using Mathematica 4.1.
%% 
%% This document uses special macros that are defined in a LaTeX 
%% package file named notebook2e.sty.  Appropriate commands for 
%% loading the package have been included in this document.
%% 
%% The LaTeX package files can be found in a directory with path:
%% 
%% /usr/site/mathematica-4.1/SystemFiles/IncludeFiles/TeX/
%% 
%% The LaTeX package used in this document may require that your 
%% LaTeX implementation support fonts other than the standard 
%% Computer Modern fonts that are used by LaTeX.  For more
%% information see the Additional Information under TeXSave
%% in the Help Browser, or at
%% 
%% http://documents.wolfram.com/v4/RefGuide/RefGuide/TeXSave.html
%% 
%% Implementation-specific instructions for installing TeX-related 
%% files can be found at
%% 
%% http://support.wolfram.com/FrontEnds/Export/TeX/
%% 
%% 

\documentclass{article}
\usepackage{notebook2e, latexsym}


\begin{document}

\Title{Spectral Filter}

\Subtitle{Scott Collis

Rice University}

The following Notepad tests the Boyd-Vandeven spectral filter as described by Levin, Iskandarani, and Haidvogel, JCP, {\bfseries 137}, 130-154 (1997).

Basically, this filter function can provides a way of ensuring that large scales are not directly affected by using the a shift or lag, {\itshape
s}.  Then the order of the falloff can be set using {\itshape p}.


\dispSFinmath{
\theta[\Mvariable{i\_},\Mvariable{s\_},\Mvariable{n\_}]\multsp :=\multsp (i-s)/(n-s);
}

\dispSFinmath{
\Muserfunction{Omega}[\Mvariable{t\_}]\multsp :=\multsp \Mfunction{Abs}[t]\multsp -1/2;
}

\dispSFinmath{
\MathBegin{MathArray}{l}
\chi[\Mvariable{t\_}]\multsp :=\multsp 1\multsp /;\multsp t\multsp =\multsp 1/2;  \\
\noalign{\vspace{0.5ex}}\chi[\Mvariable{t\_}]\multsp :=\multsp   \\
\noalign{\vspace{0.5ex}}
\hspace{1.em} \Mfunction{Sqrt}[\multsp -\log [1\multsp -\multsp 4\Muserfunction{Omega}[t]\RawWedge 2]/  \\
\noalign{\vspace{0.5ex}}
\hspace{3.em} (4\Muserfunction{Omega}[t]\RawWedge 2)]\\
\MathEnd{MathArray}
}

\dispSFinmath{
\MathBegin{MathArray}{l}
\sigma[\Mvariable{i\_},\Mvariable{s\_},\Mvariable{n\_},\Mvariable{p\_}]\multsp :=\multsp 1\multsp /;\multsp i\multsp <\multsp s;  \\
\noalign{\vspace{0.5ex}}  \\
\MathBegin{MathArray}{l}
\sigma[\Mvariable{i\_},\Mvariable{s\_},\Mvariable{n\_},\Mvariable{p\_}]\multsp :=\multsp   \\
\noalign{\vspace{0.5ex}}
\hspace{1.em} 1/2\multsp \Mfunction{Erfc}[\multsp 2\multsp \Mfunction{Sqrt}[p]\multsp \chi[\theta[i,s,n]]  \\
\noalign{\vspace{0.5ex}}
\hspace{5.em} (\Mfunction{Abs}[\theta[i,s,n]]-1/2)]\multsp /;\multsp   \\
\noalign{\vspace{0.5ex}}
\hspace{2.em} i\multsp \geq \multsp s\multsp \multsp \&\&\multsp i\multsp \leq \multsp n;\\
\MathEnd{MathArray}
\MathEnd{MathArray}
}

Plot the filter in a normalized wave number space.


\dispSFinmath{
\MathBegin{MathArray}{l}
p\multsp =\multsp 3;  \\
\noalign{\vspace{0.5ex}}  \\
n\multsp =2;  \\
\noalign{\vspace{0.5ex}}  \\
s\multsp =\multsp 0.5;  \\
\noalign{\vspace{0.5ex}}  \\
\Mfunction{Plot}[\multsp \sigma[i*n,s*n,\multsp n,p],\multsp \{i,0,1\}\multsp ]
\MathEnd{MathArray}
}

\mathGraphic{sf_gr1.eps}
\dispSFoutmath{
\SkeletonIndicator \Mvariable{Graphics}\SkeletonIndicator 
}

Now make the plot in mode space so that I can experiment with different element orders.


\dispSFinmath{
\MathBegin{MathArray}{l}
p\multsp =\multsp 5;  \\
\noalign{\vspace{0.5ex}}  \\
n\multsp =5;  \\
\noalign{\vspace{0.5ex}}  \\
s\multsp =\multsp 0.5;  \\
\noalign{\vspace{0.5ex}}  \\
\Mfunction{Plot}[\multsp \sigma[i,s*n,\multsp n,p],\multsp \{i,0,n\}\multsp ]
\MathEnd{MathArray}
}

\mathGraphic{sf_gr2.eps}
\dispSFoutmath{
\SkeletonIndicator \Mvariable{Graphics}\SkeletonIndicator 
}

Plot the derivative of the filter in spectral space.


\dispSFinmath{
f[\Mvariable{x\_}]\multsp :=\multsp \Mfunction{D}[\sigma[a,s*n,n,p],a]\multsp /.\multsp a\rightarrow x
}

\dispSFinmath{
\Mfunction{Plot}[\multsp f[x*n],\multsp \{x,0,1\}\multsp ]
}

\mathGraphic{sf_gr3.eps}
\dispSFoutmath{
\SkeletonIndicator \Mvariable{Graphics}\SkeletonIndicator 
}


\end{document}
