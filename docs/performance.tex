\documentclass[11pt]{amsart}
\usepackage[margin=1in]{geometry}
\geometry{letterpaper}
\usepackage[parfill]{parskip}     % Empty line rather than an indent
\usepackage{graphicx}
\usepackage{amssymb}
\usepackage{epstopdf}
\DeclareGraphicsRule{.tif}{png}{.png}{`convert #1 `dirname #1`/`basename #1 .tif`.png}

\title{DGM Performance Notes}
\author{S.\ Scott Collis}
\date{\today}

\begin{document}
\maketitle

\section{Notes on \texttt{DGM::Hex} Performance}
While normal Hex elements are relatively efficient because of the tensor product structure of the Legendre basis, curved Hex elements can suffer from a significant performance penalty due to the loss of orthogonality in the presence of the non-constant Jacobian.  This leads to a block-diagonal mass matrix for each element (and each component of a vector solution field).

The current approach to solve this system is to do a Cholesky (or optionally LU although Cholesky is preferred since the blocks are symmetric) decomposition and a back solve is then performed for each Galerkin projection (actually as part of the \texttt{backward\_transform} method on the Element).

There are several approaches that may help to alleviate this performance penalty
\subsection{Block solve} 
Perhaps the first thing to do is to do a multi-vector Cholesky over the vector field components.  Right now, a Cholesky back-solve is done for each \texttt{vField} component and this can be very inefficient for large vector fields.  Instead, we can implement the \texttt{project} method at the \texttt{vField} level to do a block Cholesky simultaneously over all components.  I suspect that for large numbers of field components, this may have a significant performance improvement.  The downside is that this approach still scales like $p^3$.  However, it may be that this makes the default method usable for slightly larger $p$.
\subsection{Spectral element} 
Use a \emph{spectral element} approach in which the mass matrix is collocated using a Gauss quadrature leading to a diagonal mass matrix.  One issue with this is that the quadrature is not exact therefore often requiring some form of filtering or dissipation to account for aliasing errors.

Consider the following transport equation
\[ u_{,t} + c u_{,x} - s = 0 \]
Introducing a weighting function, $w$ and integrating over the space-time domain $Q$ leads to
\[ \int_Q w \cdot ( u_{,t} + c u_{,x} - s ) \, dQ = 0 \]
We can define the mass matrix as 

\subsection{Iterative methods} 
Use an iterative solver for the block diagonals.  This might have promise at high polynomial orders as one can use a preconditioned conjugate-gradient method where the preconditioned is the spectral-element diagonal matrix.  This may help break the $p^3$ performance penalty.  Clearly a downside is that this may introduce iteration error if the mass solve tolerance is not sufficiently small.
\end{document}  