%==============================================================================
%
% Discontinuous Galerkin Method, Assignment #1
%
%  Author:   S. Scott Collis
%
%  Written:  9-28-2001
%
%  Revised:  9-28-2001
%
%==============================================================================
\documentclass[11pt]{article}
\usepackage{times}
\usepackage{fullpage}
\usepackage{comment}
%
%.... Pdf links
%
\usepackage[pdfmark,%
            colorlinks=true,%
            breaklinks=true,%
            urlcolor=blue,%
            pdfauthor={S.S.\ Collis},%
            bookmarksopen=false,%
            pdfpagemode=None]{hyperref}

\newcommand{\tecplot}{{\sf Tecplot}}
\newcommand{\dgm}{\tt dgm}

\begin{document}

\begin{center}
{\normalsize\bf Discontinuous Galerkin Exercises}\\
{\normalsize S.\ Scott Collis}\\
{\normalsize \today}
\end{center}

\section*{\normalsize\bf General Issues}

On the SGI Octane, Blasius, use CVS to checkout version {\tt rel-0-14} of the
DGM code by typing the command:
\begin{center}
\tt cvs checkout -r rel-0-14 dgm
\end{center}
This version supports the one-dimensional advection-diffusion equation as well
as the one- and two-dimensional wave equation.  The default is currently the
pure wave equation which is selected by compiling the code with the C++
compiler option {\tt -DTWOD} on.  {\em To run 1-d viscous problems you must
remove this compiler option from the Makefile!}.  Do so, them compile by
typing {\tt gmake} at the unix prompt when in the {\tt dgm/src} directory.

\medskip
\noindent {\bf Reminder: this code is currently under develoment and under no
conditions should copies be given to people outside of the Flow Physics and
Simulation research group.}

\section*{\normalsize\bf Experimentation}

I first recommend that you experiment with the working code.  If you simply
type {\tt dgm} at the unix prompt the code will generate a one-dimensional
mesh with 20 elements each of polynomial order $p=5$.  An initial condition
composed of a shifted cosine function will be set and the solution advanced in
time for 1 time unit.  Upon completion, the code generates two \tecplot\ data
files; one called {\tt ic.dat}, which is the initial condition, and {\tt
output.dat} that is the final time solution.  You can plot both of these by
reading the \tecplot\ layout file in the {\tt dgm/src} directory called {\tt
plot.lay}.  If you type {\tt dgm -help} you will see several options that can
be set through the command line.  This is currently the only way of changing
run parameters without resorting to recompilation -- this will be fixed soon.
You might try:
\begin{enumerate}
\item Changing the time step, {\tt -dt 0.01}
\item Setting a positive yet small viscosity, {\tt -nu 0.002 -dt 0.01}
\item Changing the final time, {\tt -tf 4.0}
\item Changing the polynomial order, {\tt -p 1}
\item Changing the number of elements, {\tt -ne 10}
\end{enumerate}
Once you are comfortable with the operation of the code, you can move on to
the following exercises.

\section*{\normalsize\bf Exercises}
\begin{enumerate}

\item {\bf Easy:} The current code uses fourth order Runge--Kutta time
advancement.  Theoretically we know that this method is stable for pure
convection problems as long as
\[ \Delta t \le \frac{2.81}{|\lambda_{max}|} \]
For simple numerical methods in space, like a first-order upwind difference on
a uniform mesh, we know that $|\lambda_{max}| = c/h$ where $c$ is the
convection velocity, $c=1$ in our problem, and $h = \Delta x$ is the mesh
spacing.  This leads to the familiar result that
\[ CFL \equiv \frac{c \Delta t}{h} \le 2.81 \]
for RK4.  However, it is well known that higher order spatial discretizations
have a more stringent stability requirement and discontinuous Galerkin is no
exception.  As the polynomial order is increased while keeping the number of
element fixed, you will find that smaller time-steps are required to keep the
solution stable.  Perform numerical experiments to establish the effect of
polynomial order and number of elements on the stability of the discontinuous
Galerkin method for the wave equation.  Repeat for the advection diffusion
equation.  You may want to coordinate with other students on this to help
cover the parameter space.

\item {\bf Easy:} The actual ``flow solver'' part of the code is
localized in the {\tt Domain} class, i.e. {\tt Domain.h} and {\tt Domain.cpp}.
Replace the working version of {\tt Domain.cpp} with the hacked version from
my home directory by typing
\begin{center}
{\tt cp $\sim$collis/Domain\_hack.cpp Domain.cpp}
\end{center}
This version should still compile but will generate erroneous results since I
have set the inviscid and viscous fluxes to be zero.  Where you see the
comment {\tt /* REPLACE WITH YOUR FLUX */} you should add the appropriate
definitions for both the inviscid and viscous fluxes.  I suggest that you
follow the method that we discussed in class and use the Lax-Friedrichs
inviscid flux and the Interior Penalty viscous flux.  When finished, compare
your results and code to my implementation which is in the original {\tt
Domain.cpp}.

\item {\bf Intermediate:} Implement an appropriate method in the {\tt Line}
element class to allow you to compute the $L_2$ norm of a function.  Use this
method to compute the $L_2$ error in the numerical solution and plot the
convergence of the solution with with both number of elements and polynomial
order.

\item {\bf Advanced:} While all inviscid numerical fluxes can be shown to be
identical to Lax-Friedrichs (or a simple upwind method) for the linear wave
equation, there are a variety of different viscous numerical fluxes in the
literature.  A very good review of the available methods is given by Arnold et
al. \cite{Arnold.etal:2001}.  Modify the code to support one or more of the
viscous fluxes listed in Table 3.1 of the Arnold et al. paper.  In particular,
I suggest that you first try the Baumann--Oden method.  If you get that
working then you might try the Bassi--Rebay and LDG methods.  If you have
problems with the notation in the Arnold et al. paper, don't hesitate to stop
by and I will be happy to give more information.
\end{enumerate}

\bibliographystyle{abbrv}
\bibliography{./bibs/flabbrev,./bibs/fem}
\end{document}
