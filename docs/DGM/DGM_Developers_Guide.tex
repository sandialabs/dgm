\documentclass{report}
\usepackage[T1]{fontenc}
\usepackage[latin9]{inputenc}
\usepackage{geometry}
\geometry{verbose,tmargin=1in,bmargin=1in,lmargin=1in,rmargin=1in}
\usepackage{fancyhdr}
\pagestyle{fancy}
\setcounter{secnumdepth}{3}
\setcounter{tocdepth}{3}
\usepackage{verbatim}
\usepackage{calc}
\usepackage{amsmath}
\usepackage{amssymb}
\usepackage{graphicx}
\usepackage{esint}
\PassOptionsToPackage{normalem}{ulem}
\usepackage{ulem}
\usepackage[unicode=true,pdfusetitle,
 bookmarks=true,bookmarksnumbered=false,bookmarksopen=false,
 breaklinks=false,pdfborder={0 0 1},backref=false,colorlinks=false]
 {hyperref}
\hypersetup{
 unicode=true, pdfusetitle, bookmarks=true,bookmarksnumbered=false,bookmarksopen=false, breaklinks=false,pdfborder={0 0 0},backref=false,colorlinks=true}

\makeatletter

%%%%%%%%%%%%%%%%%%%%%%%%%%%%%% User specified LaTeX commands.
\usepackage{pdflscape}
\lhead{}
\chead{DGM Developer's Guide}
\rhead{\thepage}
\lfoot{}
\cfoot{DGM Developer's Guide}
\rfoot{}

\makeatother

\begin{document}

\title{DGM Developer's Guide}


\author{S. Scott Collis, Curtis C. Ober}
\maketitle
\begin{abstract}
The DGM developer's guide is a collection of notes for developer's to
help document and educate developer's with conventions and coding practices
for the development of DGM.
\end{abstract}

\chapter{Basics}

\section{Formatting}

The basic formatting conversions are:
\begin{enumerate}
  \item Text will fit in 80 columns.  Many editing packages, webviewing,
        and other GUIs default to 80 columns.  This way text wrapping
        can be avoided.
\end{enumerate}

\section{Use of Ordinal, int, Size, LocalSize, GlobalSize, Scalar}

From Scott: Here is a quick version:

\begin{itemize}
  \item DGM::Scalar (the type held by Elements to represent the PDE solution)
  \item DGM::Real (a real number, used for coordinates and such), currently
  \item DGM::Real is DGM::Scalar but one day DGM::Scalar could be
  \item std::complex<DGM::Real> for example.
  \item DGM::Ordinal $\equiv$ DGM::LocalSize and is used to index on-node
        quantities
  \item DGM::Size $\equiv$ DGM::GlobalSize and is used to index full-scale
        quantities
\end{itemize}

Notes:
\begin{enumerate}
  \item In general, DGM::Ordinal and DGM::Size must be okay to be either
        signed or unsigned.
  \item The basic types are Ordinal and Size but we sometimes may typedefs
        (or template on) LocalSize and GlobalSize as this can make the code
        easier to decipher.
  \item All CMC::Vector are indexed on DGM::Ordinal since this must fit on
        one core.
  \item If you want a maximally 64bit version then you build with
        DGM::Ordinal = size\_t and DGM::Size = unsigned long long.
        This works great, but due to limits in Trilinos, we cannot
        enable it when this combination.
\end{enumerate}

int is only used when it *must* be (like for MPI and Epetra) and for
return values that get communicated back to the OS.  When we need too, we
use boost::numeric\_cast<int> to convert Ordinal and Size to ints and this
will catch overflows.

It is bad when int or unsigned are used directly (even locally deep in a
kernel) and I've had to tease a number of these out as they may lead to
overflows one day.

From Drew: Running ROL in DGM.  
\begin{enumerate}
  \item To choose ROL optimizers, change the {\tt opttype} parameter in {\tt root.inp} 
        to 4.
  \item The specific optimization algorithm used is determined through parameters 
        in {\tt root.rol}.  For an example of {\tt root.rol} see 
        \begin{verbatim}
           /dgm/examples/dakota/heat-dgm/heat.rol.
        \end{verbatim} 
  \item The ROL-DGM interface has the feature to mimic either DGM or PEopt {\tt root.ocs} 
        output.  This output specification is controlled through the parameter 
        {\tt output\_type} in {\tt root.inp}.  The parameter {\tt output\_type} corresponds 
        to an unsigned integer.  Independent of the value of {\tt output\_type}, the ROL 
        optimizer creates {\tt root.ocs} in ROL format.  If {\tt output\_type} is 1, then 
        we create the file {\tt root\_dgm.ocs} using DGM-style output.  If {\tt output\_type} 
        is 2, then we create the file {\tt root\_peopt.ocs} using PEopt-style.  If 
        {\tt output\_type} is 0 or greater than 2, then we only create {\tt root.ocs}.  Note 
        that in DGM style output, the ``{\tt Alpha}'' corresponds to the norm of the 
        optimization step, not the linesearch parameter size.  Similarly, for peopt-style 
        output, ``{\tt ||dx||}'' corresponds to the norm of the optimization step, not the 
        preconditioned step.  Furthermore, for PEopt-style output, we do not include 
        ``{\tt LSIter}'' or the Krylov columns because ROL manages these components internally.    
\end{enumerate}


\end{document}
