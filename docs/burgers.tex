\documentclass[12pt]{article}
\usepackage{fullpage}
\usepackage{comment}

\usepackage{amsbsy}
%%%%%%%%%%%%%%%%%%%%%%%%%%%%%%%%%%%%%%%%%%%%%%%%%%%%%%%%%%%%%%%%%%%%%%%%%%%%%%%
%
%  Macros: 
%
%  S. Scott Collis
%
%  Written: 9-5-95
%
%  Revised: 1-29-96
%
%%%%%%%%%%%%%%%%%%%%%%%%%%%%%%%%%%%%%%%%%%%%%%%%%%%%%%%%%%%%%%%%%%%%%%%%%%%%%%%
%
%  Change thesis to report for the technical report
%
\newcommand{\thesis}{thesis}
\newcommand{\eref}[1]{\mbox{\rm(\ref{#1})}}
%
% LaTeX shortcuts
%
%\renewcommand{\S}{section\ }		% This lets you change section command
\newcommand{\eqn}[1]{(\ref{#1})}	% Handy reference to equations
\renewcommand{\epsilon}{\varepsilon}	% I like the other epsilon
\newcommand{\enp}{\epsilon}		% Nonparallel epsilon
%
\newcommand{\refs}{[{\bf ?}]\marginpar{$\Longleftarrow$}}
\newcommand{\eton}{\ensuremath{e^N}}
\newcommand{\Rgas}{{\cal R}}
\newcommand{\comma}{{\, ,}}
\newcommand{\period}{{\, .}}

% For units of measure
\newcommand\dynpercm{\nobreak\mbox{$\;$dynes\,cm$^{-1}$}}
\newcommand\cmpermin{\nobreak\mbox{$\;$cm\,min$^{-1}$}}

% Various bold symbols
\providecommand\bnabla{\boldsymbol{\nabla}}
\providecommand\bcdot{\boldsymbol{\cdot}}
\newcommand\biS{\boldsymbol{S}}
\newcommand\etb{\boldsymbol{\eta}}

% For multiletter symbols
\newcommand\Real{\mbox{Re}} % cf plain TeX's \Re and Reynolds number
\newcommand\Imag{\mbox{Im}} % cf plain TeX's \Im
\newcommand\Man{\mbox{\textit{M}}}  % Mach number
\newcommand\Ren{\mbox{\textit{Re}}}  % Reynolds number
\newcommand\Prn{\mbox{\textit{Pr}}}  % Prandtl number, cf TeX's \Pr product
\newcommand\Pen{\mbox{\textit{Pe}}}  % Peclet number

\newcommand\Ai{\mbox{Ai}}            % Airy function
\newcommand\Bi{\mbox{Bi}}            % Airy function

\newcommand {\real} {I\!\!R}
\newcommand {\nat} {{I\!\!N}}
\newcommand {\half} {\mbox{$\frac{1}{2}$}}

\def\frechet{Fr\'echet\ }
\newcommand{\deq}{\raisebox{0pt}[1ex][0pt]%
{$\stackrel{\scriptscriptstyle{\rm def}}{{}={}}$}}

% For sans serif characters:
% The following macros are setup in JFM.cls for sans-serif fonts in text
% and math.  If you use these macros in your article, the required fonts
% will be substitued when you article is typeset by the typesetter.
%
% \textsfi, \mathsfi   : sans-serif slanted
% \textsfb, \mathsfb   : sans-serif bold
% \textsfbi, \mathsfbi : sans-serif bold slanted (doesnt exist in CM fonts)
%
% For san-serif roman use \textsf and \mathsf as normal.
%
\newcommand\ssC{{\mathsf{C}}} % for sans serif C
\newcommand\ssP{{\mathsf{P}}}  % for sans serif sloping P
\newcommand\ssQ{{\mathsf{Q}}}  % for sans serif bold-sloping Q

% Hat position
\newcommand\hatp{\skew3\hat{p}}      % p with hat
\newcommand\hatR{\skew3\hat{R}}      % R with hat
\newcommand\hatRR{\skew3\hat{\hatR}} % R with 2 hats
\newcommand\doubletildesigma{\skew2\tilde{\skew2\tilde{\Sigma}}}
%       italic Sigma with double tilde

% array strut to make delimiters come out right size both ends
\newsavebox{\astrutbox}
\sbox{\astrutbox}{\rule[-5pt]{0pt}{20pt}}
\newcommand{\astrut}{\usebox{\astrutbox}}

\newcommand\GaPQ{\ensuremath{G_a(P,Q)}}
\newcommand\GsPQ{\ensuremath{G_s(P,Q)}}
\newcommand\p{\ensuremath{\partial}}
\newcommand\tti{\ensuremath{\rightarrow\infty}}
\newcommand\kgd{\ensuremath{k\gamma d}}
\newcommand\shalf{\ensuremath{{\scriptstyle\frac{1}{2}}}}
\newcommand\sh{\ensuremath{^{\shalf}}}
\newcommand\smh{\ensuremath{^{-\shalf}}}
\newcommand\squart{\ensuremath{{\textstyle\frac{1}{4}}}}
\newcommand\thalf{\ensuremath{{\textstyle\frac{1}{2}}}}
\newcommand\Gat{\ensuremath{\widetilde{G_a}}}
\newcommand\ttz{\ensuremath{\rightarrow 0}}
\newcommand\ndq{\ensuremath{\frac{\mbox{$\partial$}}{\mbox{$\partial$} n_q}}}
\newcommand\sumjm{\ensuremath{\sum_{j=1}^{M}}}
\newcommand\pvi{\ensuremath{\int_0^{\infty}%
  \mskip \ifCUPmtlplainloaded -30mu\else -33mu\fi -\quad}}

%  Latin abbreviations

\newcommand{\etal}{{\em et al.\@\xspace}}
\newcommand{\etc}{{\em etc.\@\xspace}}
\newcommand{\ie}{{\em i.e., \xspace}}
\newcommand{\eg}{{\em e.g., \xspace}}
\newcommand{\apriori}{{\em a priori \xspace}}
\newcommand{\adhoc}{{\em ad hoc \xspace}}

\newtheorem{lemma}{Lemma}
\newtheorem{corollary}{Corollary}

\newcommand{\bU}{{{\bf U}}}
\newcommand{\bbU}{{\overline{\boldsymbol U}}}
\newcommand{\cbU}{{\overline{\widetilde{{\boldsymbol U}}}}}
\newcommand{\tbU}{{\widetilde{\boldsymbol U}}}
\newcommand{\hbU}{{\widehat{\boldsymbol U}}}
\newcommand{\btbU}{{\overline{\tbU}}}

\newcommand{\bvbar}{{\overline{\boldsymbol v}}}
\newcommand{\pbar}{{\overline{p}}}
\newcommand{\bF}{{\bf F}}
\newcommand{\hbF}{{\widehat\bF}}
\newcommand{\boldf}{{\boldsymbol f}}
\newcommand{\bq}{{\boldsymbol q}}
\newcommand{\bg}{{\boldsymbol g}}
\newcommand{\hbq}{{\widehat{\boldsymbol q}}}
\newcommand{\bphi}{{{\boldsymbol \phi}}}
\newcommand{\bPhi}{{{\boldsymbol \Phi}}}
\newcommand{\hphi}{{\widehat{\phi}}}
\newcommand{\bw}{{\boldsymbol w}}
\newcommand{\bv}{{\boldsymbol v}}
\newcommand{\bu}{{\boldsymbol u}}
\newcommand{\tbv}{{\widetilde{\boldsymbol v}}}
\newcommand{\hbv}{{\widehat{\boldsymbol v}}}
\newcommand{\tp}{{\widetilde{p}}}
\newcommand{\hp}{{\widehat{p}}}
\newcommand{\hu}{{\widehat{u}}}
\newcommand{\hw}{{\widehat{w}}}
\newcommand{\bL}{{\mathcal L}}
\newcommand{\tbL}{{\widetilde{\mathcal L}}}
\newcommand{\bj}{{\boldsymbol j}}
\newcommand{\bJ}{{\boldsymbol J}}
\newcommand{\bI}{{\boldsymbol I}}
\newcommand{\bk}{{\boldsymbol k}}
\newcommand{\bm}{{\boldsymbol m}}
\newcommand{\blambda}{{\boldsymbol\lambda}}
\newcommand{\bsigma}{{\boldsymbol\sigma}}
\newcommand{\tbsigma}{{\widetilde{\boldsymbol\sigma}}}
\newcommand{\bS}{{\bf S}}
\newcommand{\tbS}{{{\widetilde{\bf S}}}}
\newcommand{\bs}{{\boldsymbol s}}
\newcommand{\bx}{{\boldsymbol x}}
\newcommand{\by}{{\boldsymbol y}}
\newcommand{\bxi}{{\boldsymbol \xi}}
\newcommand{\hbs}{{\widehat{\boldsymbol s}}}
\newcommand{\tbs}{{{\widetilde{\boldsymbol s}}}}
\newcommand{\Div}{{\mbox{div\,}}}
\newcommand{\Def}{{\mbox{def\,}}}
\newcommand{\Grad}{{\mbox{grad\,}}}
\newcommand{\bG}{{\boldsymbol G}}
\newcommand{\bA}{{\bf A}} %{{\boldsymbol B}}
\newcommand{\bB}{{\boldsymbol B}}
\newcommand{\bb}{{\boldsymbol b}}

\newcommand{\tl} { \tilde }
\newcommand{\wtl} { \widetilde }
\newcommand{\who}[1] { \widehat {\overline {#1}} }
\newcommand{\ol} { \overline }
\newcommand{\ovl} {\overline}
\newcommand{\wh}{ \widehat }
\newcommand{\ou}{ \overline u }
\newcommand{\op}{ \overline p }
\newcommand{\odelta}{ \overline \Delta}
\newcommand{\oDelta}{ \overline \Delta}

\newcommand{\os} {\overline S}
\newcommand{\oS} {\overline S}

\newcommand{\pdt}[1] { \frac{\partial{#1}}{\partial t}}
\newcommand{\pdi}[2] { \frac{\partial{#1}}{\partial x_{#2}} }
\newcommand{\pdl}[2] { \frac{\partial}{\partial x_{#2}} \left({#1} \right) }
\newcommand{\pdone}[1] { \frac{\partial{#1}}{\partial x_{1}}}
\newcommand{\pdtwo}[1] { \frac{\partial{#1}}{\partial x_{2}} }
\newcommand{\pdthr}[1] { \frac{\partial{#1}}{\partial x_{3}} }
\newcommand{\pdoone} {\frac{\partial}{\partial x_{1}} }
\newcommand{\pdotwo} {\frac{\partial}{\partial x_{2}} }
\newcommand{\pdothr} {\frac{\partial}{\partial x_{3}} }
\newcommand{\pdd}[1] 
{\frac{\partial^{2}{#1}}{{\partial x_{j}}{ \partial x_{j}}} }

\newcommand{\onehalf} {\frac{1}{2}}

\newcommand{\into} { \int_\Omega }
\newcommand{\intw} { \int_{\Gamma_w} }

\newcommand{\intt} { \int_{t_0}^{t_0+T} }
\newcommand{\dto} { dt \, d\Omega}
\newcommand{\dtw} { dt \, d\Gamma}
\newcommand{\dom} { \, d\Omega }

% fluids

\newcommand{\ret}{{\Ren_{\tau}}}

% multiscale

\newcommand{\cP}{{\mathcal P}}
\newcommand{\cW}{{\mathcal W}}
\newcommand{\cV}{{\mathcal V}}
\newcommand{\cN}{{\mathcal N}}
\newcommand{\cJ}{{\mathcal J}}
\newcommand{\cT}{{\mathcal T}}

\newcommand{\bcV}{{\overline{\cV}}}
\newcommand{\tcV}{{\widetilde{\cV}}}
\newcommand{\hcV}{{\widehat{\cV}}}
\newcommand{\btcV}{{\overline\tcV}}

\newcommand{\bbu}{{\overline\bu}}
\newcommand{\cbu}{{\overline{\widetilde{\bu}}}}
\newcommand{\tbu}{{\widetilde\bu}}
\newcommand{\hbu}{{\widehat\bu}}
\newcommand{\btbu}{{\overline\tbu}}

\newcommand{\bW} {{\bf W}} %{{\boldsymbol W}}
\newcommand{\bbW}{{\overline\bW}}
\newcommand{\cbW}{{\overline{\widetilde{\bW}}}}
\newcommand{\tbW}{{\widetilde\bW}}
\newcommand{\hbW}{{\widehat\bW}}
\newcommand{\btbW}{{\overline\tbW}}

\newcommand{\bV} {{\boldsymbol V}}
\newcommand{\bbV}{{\overline\bV}}
\newcommand{\tbV}{{\widetilde\bV}}
\newcommand{\hbV}{{\widehat\bV}}
\newcommand{\btbV}{{\overline\tbV}}

\newcommand{\bbw}{{\overline\bw}}
\newcommand{\cbw}{{\overline{\widetilde{\bw}}}}
\newcommand{\tbw}{{\widetilde\bw}}
\newcommand{\hbw}{{\widehat\bw}}
\newcommand{\btbw}{{\overline\tbw}}
\newcommand{\bbk}{{\overline\bk}}

\newcommand{\bn}{{\boldsymbol n}}
\newcommand{\tnu}{\widetilde\nu}
\newcommand{\tT}{\widetilde T}
\newcommand{\trho}{\widetilde\rho}

\newcommand{\bR}{{\boldsymbol R}}
\newcommand{\bLambda}{{\boldsymbol\Lambda}}

\newcommand{\bD}{{\bf D}}%{{\boldsymbol D}}

% bold is used to produce bold face vectors in equations
%\newcommand{\bold}[1]{\boldsymbol #1}
\newcommand{\btau}{\bold{\tau}}

% other stuff

\newcommand{\figvcenter}{\hbox{}\vfil}

% DNS and LES

\newcommand{\uh}{{\hat u}}
\newcommand{\buh}{{\bf\hat u}}
\newcommand{\ph}{{\hat p}}
\newcommand{\be}{{\bf e}}
\newcommand{\fh}{{\hat f}}

% DGM

\newcommand{\lavg}{{\langle}}
\newcommand{\ravg}{{\rangle}}
\newcommand{\ljmp}{\left [}
\newcommand{\rjmp}{\right ]}
\newcommand{\bM}{{\bold M}}

%
% Adjustable parenthesis et al
%
\newcommand{\K}[1]{\left({#1}\right)}
\newcommand{\lp}{\left(}
\newcommand{\rp}{\right)}
\newcommand{\lb}{\left[}
\newcommand{\rb}{\right]}
\newcommand{\lc}{\left\{}
\newcommand{\rc}{\right\}}
\newcommand{\la}{\left\langle}
\newcommand{\ra}{\right\rangle}
\newcommand{\Dp}[3]{\langle #1,#2 \rangle_{#3}}

%
% Space-time domains
%
\newcommand{\sQ}{{\sf Q}}
\newcommand{\sP}{{\sf P}}


\begin{document}

\begin{center}
{\Large\bf Adjoint Analysis for DG Discretizations of \\[0.5ex]
Burgers Equation}\\[2ex]
S.\ Scott Collis \\
Sandia National Laboratories \\
Optimization and Uncertainty Quantification \\
Albuquerque, NM 87815-0370 \\
sscoll@sandia.gov \\[1ex]
\today
\end{center}

\section{Introduction}

The purpose of this article is to present the derivation of the adjoint
equations for a mixed hyperbolic-elliptic equation using a discontinuous
Galerkin (DG) discretization.  As a protypical example, we work with the
viscous Burgers equation in one space dimension which takes the strong form
\begin{equation} \label{e:Burgers_strong_form}
u_{,t} + (f(u))_{,x} - (\nu u_{,x})_{,x} = s
\end{equation}
subject to appropriate initial and boundary conditions.

We emphasize to the reader that the analysis presented here is formal in the
sense that we assume existence of solutions and are vague in defining the
function spaces in which the solution and dual states reside.  Such details
can be important, but detract from the main goal of the current presentation
which is to provide a succict description of the State and Adjoint methods for
DG discretizations.

%==============================================================================
%                            D I F F U S I O N
%==============================================================================

\section{Diffusion}

In this first draft, we focus the discussion only on the elliptic part of
(\ref{e:Burgers_strong_form}) since the fomulation and implementation of
elliptic operators in DG methods is less natural (and therefore more prone to
mistakes in the formulation) than are first-order convective operators.
Future versions of this document will discuss both convection and diffusion
operators.

\subsection{State Equation}
Start with the weak form (with only diffusion present) on one element
\begin{equation}
\int_{\Omega_e} w \left( u_{,t} - (\nu u_{,x})_{,x} \right) d\Omega = 0
\end{equation}
Let $f_x^v(u_{,x}) = \nu u_{,x}$ and apply DG
\begin{equation}
\int_{\Omega_e} w \left( u_{,t} - (\nu\sigma)_{,x} \right) d\Omega -
\int_{\partial\Omega_e} w \left( \hat f^v_n(\sigma^+,\sigma^-) - 
                            f^v_n(\sigma^-) \right) d\partial\Omega = 0
\end{equation}
for all $w \in {\cal V}$.  The ``jump savy'' gradient $\sigma$ is computed
using the auxillary equation
\begin{equation}
 \int_{\Omega_e} \left( \tau\sigma - \tau u_{,x} \right) d\Omega -
 \int_{\partial\Omega_e} n \tau\left( \hat u(u^+,u^-) - u^- \right) 
 d\partial\Omega = 0
\end{equation}
for all $\tau \in {\cal S}$.  The numerical fluxes are
\begin{equation}
  \hat f^v_n(\sigma^+,\sigma^-) = \frac{1}{2}\left( f^v_n(\sigma^-) - 
                                                    f^v_{-n}(\sigma^+) \right)
\end{equation}
\begin{equation}
  \hat u(u^+,u^-) = \frac{1}{2}\left( u^+ + u^- \right)
\end{equation}
On a Neumann boundary, currently I set $f^v_{-n}(\sigma^+)=-\phi_b$ and $u^+ =
u^-$.  Assuming that we have such BC's on both sides, this yields
\begin{equation}
\int_{\Omega_e} w \left( u_{,t} - (\nu\sigma)_{,x} \right) d\Omega -
\int_{\partial\Omega_e} \frac{w}{2}\left( \phi_b - n^-\nu\sigma^- \right) 
d\partial\Omega = 0
\end{equation}
for all $w \in {\cal V}$ with $\sigma$ computed using the auxillary equation
\begin{equation}
 \int_{\Omega_e} \left( \tau\sigma - \tau u_{,x} \right) d\Omega = 0
\end{equation}
for all $\tau \in {\cal S}$.

\medskip
\noindent{\bf Remarks:}
\begin{enumerate}
\item By setting the boundary condition through the numerical fluxes, we still
  retain a weighted residual formulation, but the boundary condition term has
  a factor of 1/2.
\item This could have been prevented by instead setting $\hat
  f^v_{n}(\sigma^+,\sigma^-) = \phi_b$ on the boundary.
\item One way to accomplish this is to instead set $f^v_{-n}(\sigma^+) =
  -2\phi_b + f^v_{n}(\sigma^-)$ when evaluating the numerical flux on the
  boundary.
\end{enumerate}

\subsection{Continuous Adjoint Equation}
The weak form of the state equation one element is
\begin{equation}
\int_0^T \int_{\Omega_e} w \left( u_{,t} - (\nu u_{,x})_{,x} \right) 
d\Omega = 0 \qquad \forall w\in{\cal V}
\end{equation}
Now, form the Euler-Lagrange identity by integrating by parts
\begin{eqnarray}
& \int_0^T \int_{\Omega_e} w \left( u_{,t} - 
  (\nu u_{,x})_{,x} \right) d\Omega =  \nonumber \\ 
& \int_0^T \int_{\Omega_e} u \left( -w_{,t} - 
  (\nu w_{,x})_{,x} \right) d\Omega +
  \int_{\Omega_e} \left[ w u \right]_0^T d\Omega + 
  \int_0^T\int_{\partial\Omega_e} 
  \left( u \nu w_{,n} - w \nu u_{,n} \right) d\partial\Omega_e 
\end{eqnarray}
So, the continuous adjoint equation on one element is simply
\begin{equation}
\int_0^T \int_{\Omega_e} u \left( -w_{,t} - 
  (\nu w_{,x})_{,x} \right) \, d\Omega = 0  \qquad \forall u\in{\cal V}
\end{equation}
We can now discretize this equation using DG to obtain
\begin{equation}
\int_{\Omega_e} u \left( w_{,t} + (\nu\tau)_{,x} \right) d\Omega +
\int_{\partial\Omega_e} u \left( \hat g^v_n(\tau^+,\tau^-) - 
                                 g^v_n(\tau^-) \right) d\partial\Omega = 0
\end{equation}
for all $u \in {\cal V}$.  The ``jump savy'' gradient $\tau$ is computed
using the auxillary equation
\begin{equation}
 \int_{\Omega_e} \left( \sigma\tau - \sigma w_{,x} \right) d\Omega -
 \int_{\partial\Omega_e} n \sigma\left( \hat w(w^+,w^-) - w^- \right) 
 d\partial\Omega = 0
\end{equation}
for all $\sigma \in {\cal S}$. Finally, the continuous {\em gradient} equation
is then
\begin{equation}
\int_{\partial\Omega_e} -w \phi_b \, d\partial\Omega + 
\int_{\Omega_e} \left[ w u \right]_0^T d\Omega = 0
\end{equation}

\medskip
\noindent{\bf Remarks:}
\begin{enumerate}
\item From the Euler-Lagrange equations for the continuous adjoint equation,
  we require that for diffusive flux control, $\phi_b = \nu u_{,n} = f^v_n$,
  the adjoint boundary condition is $\nu w_{,n} = 0 = g^v_n$
\item Likewise, the gradient is then given by $w$ on the boundary which means
  that $w^+=w^-$ on the boundary.  
\item In practise, I actually set the adjoint BC through the numerical flux
  which means that I (again) introduce a factor of 1/2 in the boundary term.
\item From the point of view of the continuous optimality conditions, this
  isn't a real problem in the sense that final discrete state and adjoint
  equations {\em are} consistent DG discretization of the continuous problems.
  However, they do {\em not} satisfy a discrete duality equation.
\end{enumerate}

\subsection{Discrete Adjoint Equation}
The second way to construct the adjoint equations is to construct an
Euler-Lagrange equation by beginning with the DG weak form of the state
equation.

\begin{eqnarray}
\int_{\Omega_e} w \left( u_{,t} - (\nu\sigma)_{,x} \right) d\Omega -
\int_{\partial\Omega_e} w \left( \hat f^v_n(\sigma^+,\sigma^-) - 
                            f^v_n(\sigma^-) \right) d\partial\Omega + 
\nonumber\\
 \int_{\Omega_e} \left( \tau\sigma - \tau u_{,x} \right) d\Omega -
 \int_{\partial\Omega_e} n \tau \left( \hat u(u^+,u^-) - u^- \right) 
 d\partial\Omega = 0
\end{eqnarray}
for all $w \in {\cal V}$ and $\tau \in {\cal S}$.
\begin{eqnarray}
\int_{\Omega_e} w \left( u_{,t} - (\nu\sigma)_{,x} \right) d\Omega -
\int_{\partial\Omega_e} w \left( \hat f^v_n(\sigma^+,\sigma^-) - 
                            f^v_n(\sigma^-) \right) d\partial\Omega + 
\nonumber\\
 \int_{\Omega_e} \left( \tau\sigma - \tau u_{,x} \right) d\Omega -
 \int_{\partial\Omega_e} n \tau\left( \hat u(u^+,u^-) - u^- \right) 
 d\partial\Omega =
\nonumber\\
\int_{\Omega_e} \left( -w_{,t} u + w_{,x} (\nu\sigma) \right) d\Omega -
\int_{\partial\Omega_e} w \left( \hat f^v_n(\sigma^+,\sigma^-) \right) 
d\partial\Omega + 
\nonumber\\
 \int_{\Omega_e} \left( \tau\sigma + \tau_{,x} u \right) d\Omega -
 \int_{\partial\Omega_e} n \tau\left( \hat u(u^+,u^-) \right) 
 d\partial\Omega + \int_{\Omega_e} \left[ w u \right]_0^T d\Omega =
\nonumber\\
\int_{\Omega_e} \left( -w_{,t} u + \tau_{,x} u \right) d\Omega -
 \int_{\partial\Omega_e} n \tau\left( \hat u(u^+,u^-) \right) 
 d\partial\Omega + 
\nonumber\\
 \int_{\Omega_e} \left( \tau\sigma +  \nu w_{,x} \sigma \right) d\Omega -
\int_{\partial\Omega_e} w \left( \hat f^v_n(\sigma^+,\sigma^-) \right) 
d\partial\Omega + \int_{\Omega_e} \left[ w u \right]_0^T d\Omega 
\begin{comment}
= \nonumber\\
\int_{\Omega_e} \left( -w_{,t} u + \tau_{,x} u \right) d\Omega -
 \int_{\partial\Omega_e} \tau u^- d\partial\Omega + 
\nonumber\\
 \int_{\Omega_e} \left( \tau\sigma +  \nu w_{,x} \sigma \right) d\Omega -
\int_{\partial\Omega_e} w \phi_b d\partial\Omega + 
\int_{\Omega_e} \left[ w u \right]_0^T d\Omega
\end{comment}
\end{eqnarray}
Let $\tau = -\tilde\tau\nu$, then the adjoint equation can be written as
\begin{equation}
\int_{\Omega_e} u \left( w_{,t} + (\nu\tilde\tau)_{,x} \right) d\Omega -
 \int_{\partial\Omega_e}n\nu\tilde\tau\left(\hat u(u^+,u^-)\right)
\, d\partial\Omega = 0
\end{equation}
for all $u\in{\cal V}$ and
\begin{equation}
 \int_{\Omega_e} \left( \sigma\tilde\tau - \sigma w_{,x} \right) d\Omega +
\int_{\partial\Omega_e} \frac{w}{\nu} \left( \hat f^v_n(\sigma^+,\sigma^-) 
\right) d\partial\Omega = 0
\end{equation} 
for all $\sigma\in{\cal S}$.  Finally, the {\em gradient} equation is then
\begin{equation}
\int_{\partial\Omega_e} -w \phi_b \, d\partial\Omega + 
\int_{\Omega_e} \left[ w u \right]_0^T d\Omega = 0
\end{equation}

\begin{comment}
\begin{equation}
\int_{\Omega_e} u \left( w_{,t} + (\nu\tilde\tau)_{,x} \right) d\Omega +
 \int_{\partial\Omega_e} u \left( 0 - \nu\tilde\tau \right) d\partial\Omega = 0
\end{equation}
for all $u\in{\cal V}$ and
\begin{equation}
 \int_{\Omega_e} \left( \sigma\tilde\tau - \sigma w_{,x} \right) d\Omega = 0
\end{equation} 
for all $\sigma\in{\cal S}$.  Finally, the {\em gradient} equation is then
\begin{equation}
\int_{\partial\Omega_e} -w \phi_b \, d\partial\Omega + 
\int_{\Omega_e} \left[ w u \right]_0^T d\Omega = 0
\end{equation}
\end{comment}

\medskip
\noindent{\bf Remarks:}
\begin{enumerate}
\item With averages fluxes, the continuous and discrete formulations should
  give the save solution.
\item So, due to the fact that we use a Galerkin method, the continuous and
  discrete adjoint equations are the same, as long as we set the boundary
  conditions consistently.
\end{enumerate}

%==============================================================================
%                            C O N V E C T I O N
%==============================================================================

\section{Convection}

In this section, we present a similar analysis for the convection term in
Burgers equation.  I haven't yet done a good job of merging these two sections
together. In particular, the notation is quite different between the two
sections --- sorry.

\subsection{State Equation}
Start with the Burgers equation in strong form
\[ y_{,t} + F_{,x}(y) 
%  - \left( \nu y_{,x} \right)_{,x} 
   - u_d = 0 \qquad (x,t)\in\Omega\times(0,T] \]
%
with periodic boundary conditions, where $F(y) = y^2/2$.  The Galerkin weak
form on one space-time element:
%
\[ \int_{Q_e} w  \left( y_{,t} + F_{,x}(y) - u_d
%  \left( \nu y_{,x} \right)_{,x} - f 
   \right) \,dQ = 0  \qquad \forall w\in{\cal P}_e \] 
%
Now use discontinuous Galerkin to couple elements
%
\[ \int_{Q_e} w \left( y_{,t} + F_{,x}(y) 
   % - \left( \nu y_{,x} \right)_{,x} 
   - u_d \right) \,dQ +
   \int_{P_e} w \left( \widehat F_n(y^-,y^+) - F_n(y) \right) \,dP = 0 \]
%
where $\widehat F_n(y^-,y^+)$ is the numerical flux.  If we use the Lax
Friedrichs then
%
\[ \widehat F_n(y^-,y^+) = \frac{1}{2}\left( F_n(y^-) + F_n(y^+) + 
   \ell \left( y^- - y^+ \right) \right) \]

\subsection{Linearized Burgers}
Linearize (take the \frechet derivative):
\[ \int_{Q_e} w \left( y'_{,t} + \left(y y'\right)_{,x} 
   % - \left( \nu y'_{,x} \right)_{,x} 
   - u'_d \right) \,dQ +
   \int_{P_e} w \left( \widehat F'_n(y'^-,y'^+;y^-,y^+) - 
                                F'_n(y';y) \right) \,dP  = 0 \]
%
The linearized Lax-Friedrichs flux is given by
\[ \widehat F'_n(y^-,y^+;y^-,y^+) = 
   \frac{1}{2}\left( 
     F'_n(y'^-;y^-) + F'_n(y'^+;y^+) + 
     \ell \left( y'^- - y'^+ \right) +
     \ell' \left( y^- - y^+ \right) 
   \right) 
\]
For Burgers
\[ \ell(y^-,y^+) = \max( |y^-|, |y^+| ) \]
However, we use a regularized version of Lax-Friedrichs where 
\[ \ell(y^-,y^+) = \frac{1}{2}( |y^-| + |y^+| ) \]
where we define the derivative of the absolute value function as
\begin{equation} \label{e:dabs}
d(|y|) \equiv 
\left\{ 
\begin{array}{rl} 
  -1, & y<0 \\
   0, & y=0 \\
   1, & y>0
\end{array}
\right.
\end{equation}
So that
\[ \ell'(y'^-,y'^+;y^-,y^+) = \frac{1}{2}( d(|y^-|)y'^- + d(|y^+|)y'^+ ) \]
%
Summing over both element interfaces
\[ \frac{w^-}{2}( d(|y^-|)y'^- + d(|y^+|)y'^+ )(y^- - y^+) +
   \frac{w^+}{2}( d(|y^+|)y'^+ + d(|y^-|)y'^- )(y^+ - y^-) \]
and collecting terms that multiply $y'^-$ yields
\[ \ell^* (y^- - y^+) y'^- = \frac{d(|y^-|)}{2}(w^- - w^+)(y^- - y^+)y'^- \]
and the terms multiplying $y'^+$ are
\[ \ell^* (y^+ - y^-) y'^+ = \frac{d(|y^+|)}{2}(w^+ - w^-)(y^+ - y^-)y'^+ \]

\subsection{Adjoint Equation}
Form the Lagrange identity
\begin{eqnarray}
   \int_{Q_e} \left( -w_{,t} y' - w_{,x} y y' + J'_{\tt Obs} y' \right) \,dQ
   % - w \left( \nu y_{,x} \right)_{,x} 
   + \int_{P_e} y' \widehat F^*_n(w'^-,w'^+;y^-,y^+) \,dP    && \nonumber\\
   + \int_{Q_e} \left( -w u'_d - J'_{\tt Obs} y' \right) \,dQ 
   + \int_{\Omega_e} [w y']_o^T \,d\Omega &=& 0 \nonumber
\end{eqnarray}
where the adjoint flux is
\[ \widehat F^*_n(w^-,w^+;y^-,y^+) = \frac{1}{2}\left( 
   y^- n w^- - y^- n w^+ + \ell \left( w^- - w^+ \right) 
   + d(|y^-|)(w^- - w^+)(y^- - y^+) \right) \]
%\[ y' ( y^- n w^- - y^- n w^- ) \]
Note that we can extract the adjoint equation
\begin{eqnarray} 
   \int_{Q_e} \left( w_{,t} y' + w_{,x} y y' 
   % + w \left( \nu y_{,x} \right)_{,x} 
   - J'_{\tt Obs} y' \right) \,dQ &+& \nonumber \\
   \int_{P_e} y' \left( \frac{1}{2}\left( y^- n w^- + y^- n w^+ - 
                 \ell \left( w^- - w^+ \right) \right) - y^- n w^- \right)\,dP
   &=& 0 \nonumber
\end{eqnarray}
%
So, the adjoint equation can be written as
\[ \int_{Q_e} w_{,t} y' + (w y)_{,x} y' - w y_{,x} y'  
   % + w \left( \nu y_{,x} \right)_{,x} 
   \,dQ +
   \int_{P_e} y' \left( \widehat F^*_n(w^-,w^+;y^-) - F^*_n(w;y) \right)\,dP 
   = 0 \]
where the adjoint Lax-Freidrichs flux is simply
\[ \widehat F^*_n(w^-,w^+;y^-) = \frac{1}{2}\left( y^- n w^- + y^- n w^+ - 
                                 \ell \left( w^- - w^+ \right) \right) \]
%
This numerical flux is nonconservative.  One way to interpret this is if you
add and subtract $y^+nw^+$ to the ``average'' portion of the flux then
\[ \widehat F^*_n(w^-,w^+;y^-,y^+) = \frac{1}{2}\left( 
   (y^- n w^- + y^+ n w^+) + (y^- n w^+ - y^+ n w^+) - \ell \left( w^- - w^+
\right) \right) \] which is equivalent to Lax-Friedrichs applied to the
adjoint with the addition of a source term.  If we add the source term from
both sides of an interelement boundary then
\[ \frac{1}{2} (y^l - y^r ) (w^l + w^r) \]

\medskip
\noindent{\bf Remarks:}
\begin{enumerate}
\item The discrete adjoint for Lax-Friedrichs flux is nonconservative for the
  adjoint. However it {\em is} consistent and stablizing!
\item The source term is due to the jumps in the base solution.  If we have a
  high order base solution, then this term will be very small. Likewise, if
  the base flow has no jumps then this term disappears.
\item If we use a Lax-Friedrichs flux for the adjoint then this source term
  will not be present and will result in an error in the gradient equation
  that can only be improved by refining the state solution.
\item The gradient equation is then
\[ -\int_{Q_e} w u_d' \, dQ + \int_{\Omega_e} [w y']_0^T \,d\Omega + 
   b'(w,u'_b) - J'_{\tt Obs} y' = 0 \]
\end{enumerate}

%=============================================================================
%                       E U L E R   E Q U A T I O N S
%=============================================================================

\section{Euler Equations}

The strong form of the Euler equations in conservation variables can be
written as
\begin{equation} \label{e:Euler_strong_form}
\bU_{,t} + \bF_{i,i} = \bS \qquad\mbox{in }\sQ
\end{equation}
where $\sQ= \Omega \times [0,T]$ is the space-time domain with perimeter $\sP
= \Gamma \times [0,T]$ for spatial domain $\Omega$ with
$\Gamma=\partial\Omega$.  The vector of conservative variables is given by
\begin{equation}
\bU = \left\{ \begin{array}{c} \rho \\ \rho \bu \\ \rho E \end{array} \right\}
\end{equation}
Formally, we assume that weak solutions to \eref{e:Euler_strong_form} exist
where $\bU \in \cV$ as we introduce the test functions $\bW \in \cV$ so that
the weak form of the Euler equations in $\sQ$ is given by
\begin{equation}
\int_\sQ \bW^T \left(\bU_{,t} + \bF_{i,i} - \bS \right) \,d\sQ = 0 
\qquad \forall \bW \in \cV
\end{equation}
which is the starting point for a standard Galerkin method applied to the
Euler equations.

Since we are interested in DG discretizations of the Euler equations, we
consider the following DG weak form
\begin{equation} \label{e:euler_dg}
\int_{\sQ_e} \bW^T \left(\bU_{,t} + \bF_{i,i} - \bS \right) \,d\sQ +
\int_{\sP_e} \bW^T \left( \hbF_n(\bU^-,\bU^+) - \bF_n(\bU^-) \right) \,d\sP 
= 0 \qquad \forall \bW \in \cV
\end{equation}
on the space-time element $\sQ_e$ with boundary $\sP_e$.  In this expression,
$\hbF_n(\bU^-,\bU^+)$ is a numerical flux, normal to the boundary $\sP_e$.

To construct the linearized Euler equations, we formally take the \frechet
derivative of \eref{e:euler_dg} about the state $\bU$ which yields
\begin{eqnarray}
&&\int_{\sQ_e} \bW^T \left(\bU'_{,t} + (\bA_i\bU')_{,i} - \bS \right) \,d\sQ +
\nonumber\\
&&\int_{\sP_e} \bW^T \left( \hbF'_n(\bU^-,\bU^+) - 
\bA^-_n\bU'^-) \right) \,d\sP = 0 \qquad \forall \bW \in \cV 
\end{eqnarray}
The adjoint Euler equations with DG discretization are then given by the
following Euler-Lagrange equation
\begin{eqnarray} \label{e:dg_euler_lagrange}
\int_{\sQ_e} \bW^T \left(\bU'_{,t} + (\bA_i\bU')_{,i} - \bS' \right) \,d\sQ +
\int_{\sP_e} \bW^T \left( \hbF'_n(\bU^-,\bU^+) - 
\bA^+_n\bU'^+) \right) \,d\sP &=& \nonumber\\
\int_{\sQ_e} \left(-\bW'_{,t} - \bA^T_i\bW_{,i} \right) \bU'^T \,d\sQ +
\int_{\sP_e} \bW^T \hbF'_n(\bU^-,\bU^+) \,d\sP - \int_{\sQ_e} \bW^T\bS'\,d\sQ 
&=& 0 % \qquad \forall \bW \in \cV \nonumber
\end{eqnarray}
In order to complete the adjoint formulation, we need to pay special attention
to the {\em flux} term appearing on the RHS of of \eref{e:dg_euler_lagrange}.
We will write the orginal numerical flux in the form
\begin{equation}
\hbF_n(\bU^-,\bU^+) = \frac{1}{2}\left( \bF_n(\bU^-) + \bF_n(\bU^+) - 
                      \bD(\bU^-,\bU^+;\bn) \right)
\end{equation}
The first term in this expression is simply a {\em centered} flux evaluted
using the local and adjacent states, while the second term is a {\em
  dissipation} term. All numerical fluxes for the conservations laws can be
written in this form (although in practise one may not want to implement the
flux in this form.).  In this form, the \frechet derivative of the numerical
flux can be written as
\begin{eqnarray}
\hbF'_n(\bU'^-,\bU'^+;\bU^-,\bU^+,\bn) &=& 
  \frac{1}{2} \biggl( \bA_n(\bU^-)\bU'^- + \bA_n(\bU^+)\bU'^+ - 
\nonumber\\ && \,\,\quad
  \bD'(\bU'^-,\bU'^+;\bU^-,\bU^+,\bn) \biggr)
\end{eqnarray}
With this, the linearized Euler equations with LF flux can be written out in
full as
\begin{eqnarray}
&&\int_{\sQ_e} \bW^T \left(\bU'_{,t} + (\bA_i\bU')_{,i} - \bS \right) \,d\sQ +
\nonumber\\
&&\int_{\sP_e} \bW^T \Biggl( \frac{1}{2} \biggl( \bA_n(\bU^-)\bU'^- + 
  \bA_n(\bU^+)\bU'^+ - \nonumber\\ &&
  \qquad\qquad\bD'(\bU'^-,\bU'^+;\bU^-,\bU^+,\bn) \biggr) - 
  \bA^-_n\bU'^- \Biggr) \,d\sP = 0 \qquad \forall \bW \in \cV 
\end{eqnarray}

Now, all we have to do is convert this expression to an equivalent adjoint
flux so that the adjoint equation can be written as
\begin{eqnarray} \label{e:Euler_adjoint}
&&\int_{\sQ_e} \bU'^T \left(-\bW'_{,t} - \bA^T_i\bW_{,i} \right) \,d\sQ + 
\nonumber\\
&&\int_{\sP_e} \bU'^T \frac{1}{2}\left( \bA^T_n(\bU^-) \bW^- - 
                                        \bA^T_n(\bU^-) \bW^+ -
                               \bD^*(\bW^-,\bW^+;\bU^-,\bU^+,\bn) \right) 
                               \,d\sP = 0
\end{eqnarray}
where we have explicitly written the adjoint of the {\em central} part of the
numerical flux, but have used the notation $\bD^*$ to denote the adjoint of
the {\em dissipation} component of the numerical flux.  This portion depends
on the particular choice of numerical flux utilized and we consider the
Lax-Friedrichs flux in the next section.

\subsection{Lax-Freidrichs Dissipation}

The dissipation term for the Lax-Freidrichs flux is given by
\begin{equation}
\bD_{LF} \equiv \ell_{max}(\bU^-,\bU^+;\bn)\left( \bU^+ - \bU^- \right)
\end{equation}
where $\ell_{max}(\bU^-,\bU^+;\bn)$ depends on the eigenvalues of the Euler
Jacobians, $\bA_n(\bU^-)$ and $\bA_n(\bU^-)$.  Formally taking the \frechet
derivative yields
\begin{equation}
\bD'_{LF} \equiv \ell_{max}(\bU^-,\bU^+;\bn)
                 \left( \bU'^+ - \bU'^- \right) +
                 \ell'_{max}(\bU'^-,\bU'^+;\bU^-,\bU^+,\bn)
                 \left( \bU^+ - \bU^- \right)
\end{equation}
The first term is straightforward to compute while the second can be
challenging since $\ell_{max}$ is often not a differentiable function of
$\bU^-$ and $\bU^+$.  While one can attempt to ignore this term, experience
with Burgers equation suggests that this is not a tenable approach, especially
when using low polynomial orders (i.e.\ FV methods) or when additional
linearizations are required (such as Newton methods applied to the
optimization problem.)

Focusing on the first term, the adjoint is given by
\begin{equation}
\bD^*_{LF_1}  = \ell_{max}(\bU^-,\bU^+;\bn)\left( \bW'^+ - \bW'^- \right)
\end{equation}
For the second term, we begin with the definition
\begin{equation}
\ell_{max} = \max_{\bU\in[\bU^-,\bU^+]} \, \max_{m\in[1,4]}
             \left( |\lambda_m| \right)
\end{equation}
where $\lambda_m$ are the eigenvalues of the Euler Jacobians, $\bA_n(\bU)$, in
the $n$-direction.  For the Euler equations in two spatial dimensions, the
eigenvalues are
\begin{equation}
\blambda = \left\{ \begin{array}{c} u_n - c \\ u_n \\ u_n \\ u_n + c
\end{array} \right\}
\end{equation}
where $c$ is the local sound speed.  Typically, the maximum value will always
be associated with one of the two acoustic modes depending on the sign of
$u_n$.  That means that the variation of $\ell_{max}$ will require the
variation of both $u_n$ (which is not difficult) 
\[ u_n = \frac{m_1 n_1 + m_2 n_2}{\rho} \]
\[ u'_n = -\rho'\frac{u_n}{\rho} + \frac{ m'_1 n_1 + m'_2 n_2 }{\rho} \]
\begin{equation}
\frac{\partial u_n}{\partial\bU} = \frac{1}{\rho}\left\{\begin{array}{cccc}
-u_n & n_1 & n_2 & 0 
\end{array}\right\}
\end{equation}
and the variation of $c$ (which is more cumbersome)
\begin{eqnarray}
c&=&\sqrt{ (\gamma-1) \left( E + \frac{p}{\rho} - 
            \onehalf( u^2 + v^2 )\right) } \nonumber\\
 &=&\sqrt{ (\gamma-1) \left( \frac{\rho E + p}{\rho} - 
                             \frac{1}{2\rho^2}( m_1^2 + m_2^2 ) \right) }
\end{eqnarray}
\begin{eqnarray} 
p &=& (\gamma-1)\left( \rho E - \frac{\rho}{2}(u^2 + v^2)\right) \nonumber\\
  &=& (\gamma-1)\left( \rho E - \frac{1}{2\rho}(m_1^2 + m_2^2) \right)
\end{eqnarray}
\begin{equation}\frac{\partial p}{\partial \bU} = 
(\gamma-1)\left\{
\begin{array}{cccc}
\frac{1}{2\rho^2}(m_1^2 + m_2^2) & -\frac{m_1}{\rho} & -\frac{m_2}{\rho} & 1 
\end{array}
\right\}
\end{equation}
\begin{eqnarray}
\frac{\partial c}{\partial \bU} &=& 
\left\{
\begin{array}{cccc}
\left(-\frac{c}{\rho}+\frac{\gamma(\gamma-1)\rho E}{2c\rho^2}\right) &
-\frac{\gamma(\gamma-1)m_1}{2c\rho^2} &
-\frac{\gamma(\gamma-1)m_2}{2c\rho^2} &
\frac{\gamma(\gamma-1)}{2c\rho}
\end{array}
\right\} \nonumber\\
%\frac{\partial c}{\partial \bU} 
&=& 
\frac{\gamma(\gamma-1)}{2c\rho^2}
\left\{
\begin{array}{cccc}
\left(\frac{-2\rho c^2}{\gamma(\gamma-1)}+\rho E\right) & -m_1 & -m_2 & \rho
\end{array}
\right\}
\end{eqnarray}

Now, because $c$ is always positive, it turns out that the maximum eigenvalue
(in absolute value) can always be expressed as $|u_n|+c$.  Likewise, since the
LF flux also takes the maximum over both the local and adjacent states, we can
introduce a {\em smoothed} LF-flux of the form
\begin{equation}
\ell_{max} = \onehalf\left( |u^-_n| + c^- + |u^+_n| + c^+ \right)
\end{equation}
so that
\begin{equation} \label{e:lmax_prime}
\ell'_{max} = 
  \onehalf\left( d(|u^-_n|)\frac{\partial u_n}{\partial\bU}(\bU^-) +
  \frac{\partial c}{\partial \bU}(\bU^-)\right)\bU'^- +
  \onehalf\left( d(|u^+_n|)\frac{\partial u_n}{\partial\bU}(\bU^+) +
  \frac{\partial c}{\partial \bU}(\bU^+)\right)\bU'^+
\end{equation}
where the derivative of the absolute value function is defined in
\eref{e:dabs}.  With \eref{e:lmax_prime} we can now complete the definition of
the adjoint {\em smoothed} LF-flux.

Focusing on the second term, the adjoint is given by
\begin{eqnarray}
\bD^*_{LF_2} &=&  
  \frac{1}{2}\left( d(|u^-_n|)\frac{\partial u_n}{\partial\bU}(\bU^-) +
  \frac{\partial c}{\partial \bU}(\bU^-)\right)^T (\bU^+-\bU^-)^T \,\bW^- 
  \nonumber\\  &-& 
  \frac{1}{2}\left( d(|u^-_n|)\frac{\partial u_n}{\partial\bU}(\bU^-) +
  \frac{\partial c}{\partial \bU}(\bU^-)\right)^T (\bU^--\bU^+)^T \,\bW^+
\end{eqnarray}
Note the minus sign that is required due to the change in norm in the second
term.  Let's define the rank-one matrix
\begin{equation}
 {\boldsymbol\ell}^*_{max}(\bU^-,\bU^+;\bn) \equiv 
  \frac{1}{2}\left( d(|u^-_n|)\frac{\partial u_n}{\partial\bU}(\bU^-) +
  \frac{\partial c}{\partial \bU}(\bU^-)\right)^T (\bU^--\bU^+)^T
\end{equation}
So the total adjoint of the LF dissipation term is
\begin{eqnarray}
\bD^*_{LF} = \left( \ell_{max}(\bU^-,\bU^+;\bn) + 
             {\boldsymbol\ell}^*_{max}(\bU^-,\bU^+;\bn) \right) 
             (\bW^+ - \bW^-)
\end{eqnarray}
{\em SSC:  Need to verify all signs!}

\medskip
\noindent{\bf Remarks:}
\begin{enumerate}
\item For systems, the adjoint of the dissipation term becomes a matrix
  operator (albeit of rank-one).
\item Just like Burgers, I have introduced a {\em smoothed} Lax-Friedrichs
  flux.  This flux may or may NOT be entropy stable and this needs to be
  established.
\item This adjoint flux an be readily implemented (although all terms need to
  be checked for correctness).
\item Equation \eref{e:Euler_adjoint} is the exact {\em discrete} adjoint for
  the Euler equations and can (in principle) be implemented for any numerical
  flux.
\end{enumerate}

\subsection{Boundary Conditions}

\subsubsection{State Boundary: $\Gamma_s$}

For a state boundary we prescribe the $\bU^+$ state as
\begin{equation}
\bU_s = \left\{ \begin{array}{c} 
\rho_s \\
(\rho u)_s \\
(\rho v)_s \\
(\rho E)_s \\
\end{array} \right\}
\end{equation}
so that the flux term on the wall boundary is given by
\begin{equation}
\int_{\sP_s} \bW^T \left( \hbF_n(\bU^-,\bU_s) - \bF_n(\bU^-) \right) \,d\sP
\end{equation}{\em Need more here\dots}

\subsubsection{Wall Boundary: $\Gamma_w$}
For a wall boundary with wall transpiration velocity $g$, we prescribe a wall
state as
\begin{equation}
\bU_w = \left\{ \begin{array}{c} 
\rho^- \\
\tilde m_1 \equiv n_2 ( m^-_1 n_2 - m^-_2 n_1 ) + n_1 g \\
\tilde m_2 \equiv n_1 ( m^-_2 n_1 - m^-_1 n_2 ) + n_2 g \\
(\rho E)^- - \frac{1}{2\rho^-}\left( (m^-_1)^2 - (\tilde m^-_1)^2 + 
                                (m^-_2)^2 - (\tilde m^-_2)^2 \right)
\end{array} \right\}
\end{equation}
so that the flux term on the wall boundary is given by
\begin{equation}
\int_{\sP_w} \bW^T \left( \hbF_n(\bU^-,\bU_w) - \bF_n(\bU^-) \right) \,d\sP
\end{equation}
Note that in this expression $\bU_w$ depends on $\bU^-$ which
changes the structure of the adjoint to
\begin{equation}
\int_{\sP_e}  \frac{1}{2}\left( \bU'^T \bA^T_n(\bU^-) \bW^- +
                                \bU'^R_w \bA^T_n(\bU^-) \bW^- -
                               \bD^*(\bW^-,\bW^+;\bU^-,\bU^-_w,\bn) \right) 
                               \,d\sP
\end{equation}
{\em This still needs work\dots}
\end{document}

%==============================================================================
%                               T H E   E N D 
%==============================================================================

%
% perhaps this is not useful here...
%
\begin{comment}
One way to construct the eigenvalues of the Euler Jacobian is via a similarity
transformation of the form
\begin{equation}
{\boldsymbol\Lambda} = {\bf L}\bA_n{\bf L}^{-1}
\end{equation}
where the columns of ${\bf L}$ are the eigenvectors of $\bA_n$ and the
diagonal entries of ${\boldsymbol\Lambda}$ are the associated eigenvalues.
\end{comment}
